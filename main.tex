\documentclass{report}

%--------------------------------------------------- PAGE STYLE 
\usepackage[utf8]{inputenc} 		% Encoding
\usepackage[T1]{fontenc} 			% To write in French
\usepackage[english]{babel} 			% To write in English
\usepackage{fancyhdr} 				% Fancy style
\usepackage[left=1.7cm,right=1.7cm,top=1.7cm,bottom=1.7cm]{geometry}
\usepackage[titletoc]{appendix}	   	% Better appendix management
\usepackage[absolute]{textpos}
\usepackage{booktabs}
\usepackage{etoolbox}					% Manage toclof lines
\usepackage[bottom]{footmisc}		% Throw footnotes to the very bottom of the page

\usepackage{expdlist}
\newenvironment{itemize*}%			% Creates a custom itemize environment, called by itemize* 
  {\begin{itemize}%
    \setlength{\itemsep}{0pt}%		% The interline is reduced so that we don't waste space
    \setlength{\parskip}{0pt}}%
  {\end{itemize}}
  
\renewcommand{\rmdefault}{phv}	% Use Helvetica as default font for the whole document

\setlength{\parindent}{0pt}	      	% Remove auto-ident

\setcounter{secnumdepth}{3} 		% default value for 'report' class is "2"

%--------------------------------------------------- COOL CHAPTER NUMBERING
\usepackage{titlesec, blindtext, color}

\titlespacing*{\chapter}{0pt}{0pt}{20pt}	% Remove space used by the "chapter" header

\definecolor{black}{gray}{0}
\newcommand{\hsp}{\hspace{20pt}}
\titleformat{\chapter}[hang]{\LARGE\bfseries}{\thechapter\hsp\textcolor{black}{|}\hsp}{0pt}{\LARGE\bfseries}

%--------------------------------------------------- BIBLIOGRAPHY 
\usepackage{natbib}

%--------------------------------------------------- MATHS STUFF 
\usepackage{float}						% For float
\usepackage{amsmath,amscd,amssymb,epsfig}  %sci equations
\usepackage{matlab-prettifier}      % MATLAB codes
\usepackage[numbered,framed]{mcode} % For MATLAB code
\usepackage{verbatim}               	% to quote code
\usepackage{listings}               	% Include the listings-package
\usepackage{lipsum}					% Cite code
\usepackage{tocloft}					% Different codes inputs manager (python, SciLab, ...)

%--------------------------------------------------- INCLUSIONS 
\usepackage{array}						% Required for tabulars 
\usepackage{graphicx} 				% Required for the inclusion of images
\usepackage[format=plain,
			font={small},
			labelfont={bf,it},
			textfont=it]{caption}   		% To include captions (small, figure in bold, all text in italic)
\usepackage{enumitem}              	% To make lists
\usepackage{url}                    		% to quote URLs
\usepackage{pdfpages}			    % To include PDFs (annexes)
\usepackage{hyperref}					% To make references 
\usepackage{movie15}					% To include gifs
\usepackage{makecell}				% Line breaks in tabular
\usepackage{adjustbox}				% Adjustable boxes to fit tabulars
\usepackage{circuitikz}					% To include circuits
\usepackage{tikz}
\newcommand{\tikzmark}[1]{\tikz[overlay, remember picture] \coordinate (#1);}

\renewcommand\theadalign{cc}
\renewcommand\theadfont{\bfseries}
\renewcommand\theadgape{\Gape[0.5pt]}
\renewcommand\cellgape{\Gape[0.5pt]}
\renewcommand\cellalign{cc}
\renewcommand\cellgape{\Gape[1pt]}

%--------------------------------------------------- TODOs / DRAFT MODE
 
\newif\ifdraft
\drafttrue
\draftfalse

\ifdraft
\usepackage[colorinlistoftodos]{todonotes}
\else
\usepackage[disable]{todonotes}
\fi

%--------------------------------------------------- CUSTOM COMMANDS
\newcommand{\ts}{\textsuperscript} 	% Simplifies superscript
\newcommand{\tsb}{\textsubscript}  	% Simplifies subscript


\newcommand{\listequationsname}{List of Equations}		% Create list of equations
\newlistof{myequations}{equ}{\listequationsname}
\newcommand{\myequations}[1]{%
  \refstepcounter{myequations}%
  \addcontentsline{equ}{myequations}
    {\protect\numberline{\theequation}#1}\par%
}
\makeatletter
	\preto\chapter{\addtocontents{equ}{\protect\addvspace{10\p@}}}%	% Set skips between chapters
\makeatother
\setlength{\cftmyequationsnumwidth}{2.5em} % Width of equation number in List of Equations
\setlength{\cftmyequationsindent}{1.5em}   % Indent of list of equation 

\hbadness=99999  % or any number >=10000
\hfuzz=3pt


\begin{document}

%---------------------------------------------------------------------
%	TITLE PAGE
%---------------------------------------------------------------------

\begin{titlepage}

\newcommand{\HRule}{\rule{\linewidth}{0.5mm}} % horizontal lines

\center % Center everything on the page

\includegraphics[scale=0.4]{logo.png} \\[3cm] % Logo

	%	HEADING 

\textsc{\large International Master Degree in Electro-Acoustics, 2\ts{nd} Year internship report}\\[0.4cm] % Major heading 
\textsc{\normalsize Academic year 2017 - 2018}\\[0.7cm] % Minor heading 
\textsc{\normalsize Company: Klippel GmbH}\\[0.7cm] % Minor heading 

	%	TITLE 

\HRule \\[0.4cm]
 { \huge \bfseries Near-Field Scanner: measurement optimization}\\[0.3cm] 
\HRule \\[1.5cm]

	% DATE
    
{\large \today}\\[2cm] 

	%	AUTHORS - SUPERVISORS

\begin{minipage}{0.4\textwidth}
\begin{flushleft} \large
\emph{Author:}\\ % Author(s)
Chlo\'{e} \textsc{Corno}

\end{flushleft}
\end{minipage}
~
\begin{minipage}{0.4\textwidth}
\begin{flushright} \large
\emph{Referees:} \\ % Supervisor(s) 
\textit{Academic:} Manuel Melon \\
\textit{Company:} Christian Bellmann \\

%\emph{Reviewer:} \\ % Reviewer Name

\end{flushright}
\end{minipage}\\[2cm]
 
\todo[inline, color=red]{DISABLE DRAFT MODE!!!!!!!!!!!!!!!!!!!!!!!!!!!!!!!!!!!!!!!!!!!!!!}
\vfill % Fill the rest of the page with white

\end{titlepage}

%---------------------------------------------------------------------
%	LIST OF TODOS - THANKS
%---------------------------------------------------------------------




%If in draft mode Acknowledgements will not be shown, but the list of todos will
\ifdraft
\listoftodos
%If not in draft mode, Acknowledgements are shown and list of tods is removed
\else
\thispagestyle{empty}
%\begin{center}
%{\Large\textbf{Acknowledgements}} 
%\end{center}
%\vspace{1cm}
%I would first like to thanks my supervisor Christian Bellmann for all the help and support he provided me during these 5 months. 
%
%\vspace{0.5cm}
%
%I want to particularly thanks the Université du Mans and all the professors and supervisors for giving me the opportunity to follow the Internal Master Degree in ElectroAcoustics and allowing me to work in a domain that passionate me. 
%
%\vspace{0.5cm}
%
%Finally, I want to express my profound gratitude to my family, who supported me during my studies. I would've never made it without you. Thank you.

\vfill % Fill the rest of the page with white
\fi	% End of condition

%---------------------------------------------------------------------
%	TABLE OF CONTENTS - LIST OF FIGURES
%---------------------------------------------------------------------
\pagenumbering{alph} % alpha page numbering for ToC and LoF

\clearpage
	% Change the name of table of contents and other lists
\renewcommand\contentsname{Table of contents}
\tableofcontents
\addtocontents{toc}{~\hfill\textbf{Page}\par} % Adds "page" on top of page number

\clearpage
\listoffigures
\addcontentsline{toc}{chapter}{List of Figures}

\clearpage
\listofmyequations
\addcontentsline{toc}{chapter}{List of Equations}

%--------------------------------------------------------------
%	CODE SAMPLES
%--------------------------------------------------------------

	% IMAGE WITHOUT FLOAT 
    
%\begin{center}
%\includegraphics[width=\textwidth]{name_of_figure} 
%\captionsetup{hypcap=false} 
%\captionof{figure}{some caption for the figure} 
%\label{fig:name_of_label}
%\end{center}

	% ADD CONTENT TO ToC 
    
%\addcontentsline{toc}{[type]}{[Name]}
% type : chapter section subsection ...?
% Name : introduction, ...?


%--------------------------------------------------------------
%	INTRODUCTION
%--------------------------------------------------------------

\chapter*{Introduction}
\addcontentsline{toc}{chapter}{Introduction}
\pagenumbering{arabic} % Arabic page numbering for text corpus


"\textit{What defines a good measurement of a speaker?}" is a common rhetorical question for anyone working in the field of electroacoustics. While purely electrical measurements such as the impedance are quite immune to noise, they do not allow to fully characterize a driver, and acoustical measurements must be conducted. \\
A popular acoustical measurement for loudspeakers is the directivity: allowing to see how a driver radiates at a given angle, this measurement is usually done using one or an array of microphones, and by placing the Device Under Test on a turntable. By rotating the Device Under Test and measuring its SPL response for several angles, the directivity of the device is measured. \\

Although this method is widely used, it presents several drawbacks:
\begin{itemize}
\item Resolution: the higher the angular resolution, the better the measurement but the longer it takes. 
\item Accuracy: a good measurement must ideally be performed in an anechoic room, and in the far field of the Device to avoid any near field effects such as diffraction and interferences.
\item Price: the materials required such as an anechoic room or a performant microphone array can either very expensive or difficult to access. 
\end{itemize}

Several solutions are available to ease off these limits, such as the Klippel Near Field Scanner measurement device, the application of Room Compensation functions to measurements; or the simulation of free field condition by measuring in an open space. In this report,  different method of assessing the sound field emitted by a loudspeaker will be explored, with a main accent on the Near Field Scanner. \\


This document will cover different points, starting with a preface presenting the company, the context and aim of this work, and a literature review. Then the different measurements and results obtained during this work will be presented thanks to three topics: measurement of systems mounted in baffle, optimization of the measurement time of the Near Field Scanner, and compensation of measurements in bad rooms such as workshops. \\

The following work have been conducted during a 5-months internship at the company Klippel GmbH in Dresden, Germany. This internship is mandatory to graduate from the International Master Degree in Electro-Acoustics (IMDEA) at the Université du Mans, in France. \\

%--------------------------------------------------------------
%	PRESENTATION OF THE COMPANY ,  MISSION , LITTERATURE REVIEW
%--------------------------------------------------------------
\chapter{Preface}

	% COMPANY 
    
\section{Presentation of the company}

\begin{minipage}{0.35\textwidth}
\centering
	\includegraphics[width=0.7\textwidth]{Preface/logo_klippel} 
    \captionsetup{hypcap=false} 
	\captionof{figure}{Logo of Klippel GmbH} 
	\label{fig:lolo_klippel}
\end{minipage}
\begin{minipage}{0.65\textwidth}
Founded in 1997 by Prof. Dr. Wolfgang Klippel from the T.U. Dresden, Klippel GmbH is a company specialized in the development of test equipment dedicated to electroacoustic transducers and audio systems, for both industry and laboratories. Thanks to lots of research and development, the company has created more accurate physical models for transducers, micro-speakers and headphones, valid for both small and large signals.  \\
Over the years, Klippel GmbH has acquired a huge notoriety for their innovative products but also thanks to many conferences and trainings. Their equipment is now used world-wide for measuring, testing, and designing loudspeakers.
\end{minipage}


	% CONTEXT, AIM OF THE WORK
    
\section{Context and aim of the work}

\begin{minipage}{0.6\textwidth}
The Near Field Scanner (NFS) is a Klippel product, shown in figure \ref{fig:nfs}, performing 3D acoustic measurements of direct sound radiated in the near field of various audio products (laptop speakers, line arrays, ...). The direct sound can then be determined at any distance and angle outside the scanning surface, allowing to estimate the far field behavior of the source. \\

The particularity of this product is that the Device Under Test (DUT) is at a fixed position, and the microphone moves thanks to a robotic arm. This allows to ensure a constant interaction between the DUT and the acoustical boundaries, such as the diffraction from the edges of a turntable. By using the field separation method described in section \ref{sec:FieldSep}, the contributions from the boundaries can be removed from the measurement. 
\end{minipage}
\begin{minipage}{0.4\textwidth}
\begin{center}
	\includegraphics[scale=0.13]{Preface/NFS2} 
    \captionsetup{hypcap=false} 
	\captionof{figure}{Klippel Near Field Scanner} 
	\label{fig:nfs}
\end{center}
\end{minipage}\newline

This work has three main guidelines, which are the measurement of a speaker mounted on a baffle using the NFS and the enhancement of the measurement grid; the optimization of measurement time by exploiting symmetry conditions on the complex coefficients of the spherical wave expansion; and the application of room compensation on non-anechoic measurements.\\

The NFS baffle measurement aims to create an application note on how to perform a near field scanning of a loudspeaker mounted on a baffle and evaluate the baffle's influence on the measurement. To do so, the required tasks are the assembly and critical evaluation a baffle prototype, and the optimization of the measurement grid.\\
The measurement time optimization part aims to reduce the scanning effort of the NFS by assuming symmetry conditions on the DUT 's geometry and applying these conditions on the complex coefficients of the spherical wave expansion (see equation \ref{eq:sphwvexp}). \\
Finally, the room compensation part aims to experiment on Room Compensation solutions,  allowing to perform fast non-anechoic measurements, and to create an application note on how to correct non-anechoic measurements.\\

There is no particular price constraints or technical constraints associated with this project, as the required hardware and software is already available. 

\newpage

	% LITERATURE REVIEW
\section{Literature review}

Performing measurements of a loudspeaker on a baffle can lead to many problems, mainly caused by the diffraction at the edges of the baffle, and the influence of the room on the measurement \cite{LIS}.\\
The Klippel Near Field Scanner uses the spherical wave expansion to determine the 3D sound pressure output of a DUT at any point in its far field, by performing a measurement in the nearfield of the DUT and solving the wave equation in spherical coordinates in order to apply the field separation method described by G. Weinreich and tested by M. Melon in the ASA publication \textit{Evaluation of a method for the measurement of subwoofers in usual rooms} and AES publication \textit{Comparison of four subwoofer measurement techniques}  \cite{melon1, melon2}. The complete theory behind this method is beyond the scope of this report, and will only be presented succinctly \citep[see][sect.~3]{aeshs} \citep[see][chap.~6]{Fourier}. 

\subsection{Field Separation}
\label{sec:FieldSep}

The solution of the three-dimensional wave equation in spherical coordinates can be described by a spherical wave expansion given by equation \ref{eq:sphwvexp} \cite{aeshs, train8}. The first expansion term, with the Hankel function of the second kind $h^{(2)}_{n}$, represents the outgoing waves; while the second expansion term, with the Hankel function of the first kind, represents the incoming waves. The Hankel functions contains the frequency dependence $e^{\pm j\omega t}$. The $Y_{n}^{m}(\theta, \phi)$ terms represents the spherical harmonics and describes the dependency on azimuthal angle $\phi$ and polar angle $\theta$. Finally, the terms $c_{n,m}^{out}(j \omega)$ and $c_{n,m}^{in}(j \omega)$ are weighting coefficients.


\begin{equation}
P(j\omega, r, \theta, \phi)  = \sum_{n=0}^{N}\sum_{m=-n}^{n}c_{mn}^{out}(j \omega)\cdot h_{n}^{(2)}(kr)\cdot Y_{n}^{m}(\theta, \phi)  + \sum_{n=0}^{N}\sum_{m=-n}^{n}c_{mn}^{in}(j \omega)\cdot h_{n}^{(1)}(kr)\cdot Y_{n}^{m}(\theta, \phi)
\label{eq:sphwvexp}
\end{equation}
\myequations{3D wave equation in spherical coordinates}

The free-field transfer function of a loudspeaker can be expressed by only considering the direct sound (outgoing wave) in equation \ref{eq:sphwvexp} and is expressed in equation \ref{eq:ffHf} \cite{aeswb}.

\begin{equation}
\begin{split}
H_{free}(f, \mathbf{r}) & = \sum_{N(f)}^{n=0}\sum_{n}^{m=-n}C_{mn}(f)\cdot h_{n}^{(2)}(kr)Y_{n}^{m}(\theta ,\phi )
\\ 
& = \mathbf{c}(f)\cdot \mathbf{b}(f,\mathbf{r})
\end{split}
\label{eq:ffHf}
\end{equation}
\myequations{Free-field transfer function of a loudspeaker in spherical coordinates}

Where $\mathbf{b}(f,\mathbf{r})$ represents the solution of the wave equation in spherical coordinates, and $\mathbf{c}(f)$ represents the weighting coefficients. The Hankel function of the second kind $h_{n}^{(2)}(kr)$ describes the radial dependency of the wave radiating from the near field to the far field. \\
For non-anechoic measurements, an additional expansion term is required to describe the influence of external sound sources (room reflections, resonances) that are described by the Bessel function $j_{n}(kr)$, leading to the general solution for the sound field measured in a non-anechoic measurement given in equation \ref{eq:ffHt}  \cite{aeswb}.

\begin{equation}
\begin{split}
H_{meas}(f, \mathbf{r}) &= \sum_{n=0}^{N(f)}\sum_{m=-n}^{n}\left [ C_{mn}^{out}(f)\cdot h_{n}^{(2)}(kr )\cdot Y_{n}^{m}(\theta ,\phi )+C_{mn}^{in}(f)\cdot j_{n}(kr)\cdot Y_{n}^{m}(\theta ,\phi ) \right ] \\
&= \mathbf{c}^{out}(f)\mathbf{b}(f,\mathbf{r})+\mathbf{c}^{in}(f)\mathbf{b}_{room}(f,\mathbf{r})
\end{split}
\label{eq:ffHt}
\end{equation}
\myequations{General solution for sound field measured in non-anechoic environments}

Where $\mathbf{c}^{out}(f)$ represents the outgoing coefficients and $\mathbf{c}^{in}(f)$ the incoming coefficients. \\
If the pressure is measured on 2 hemispherical surfaces in front of the baffle, the sound field can be expressed as equation \ref{eq:ffHt} for each surface. This can be expressed in matrix form by equation \ref{eq:holo_fieldsep} \cite{fieldsep}.

\begin{equation}
\begin{split}
&H_{1} = \mathbf{c}_{1}^{out}\mathbf{b}_{1}+\mathbf{c}_{1}^{in}\mathbf{b}_{room_{1}} \\
&H_{2} = \mathbf{c}_{2}^{out}\mathbf{b}_{2}+\mathbf{c}_{2}^{in}\mathbf{b}_{room_{2}}
\end{split}
\label{eq:holo_fieldsep}
\end{equation}
\myequations{System of equation of measured sound field}

Each set of base functions $\mathbf{b}$ and $\mathbf{b}_{room}$ contains the elementatry functions. The subscripts $1$ and $2$ refer to the two scanning surfaces. This system of equations can be assembled and expressed as equation \ref{eq:holo_matrix} \cite{fieldsep}.

\begin{equation}
H_{m} = B_{h}\mathbf{c}_{m}
\label{eq:holo_matrix}
\end{equation}
\myequations{Matrix form of measured sound field}

Where $ H_{m} $ represents the vector of measured pressures and $ \mathbf{c}_{m} $ the matrix of outgoing and ingoing coefficients. After retrieving the coefficient matrix $ \mathbf{c}_{m} $ by performing the inversion of equation \ref{eq:holo_matrix}, the outgoing field from the primary source can be retrieved and is expressed in equation \ref{eq:holo_solve}; where $ b(f,\mathbf{r})$ is the matrix containing the outgoing elementary spherical harmonics $Y_{n}^{m}(\theta ,\phi )$ \cite{fieldsep}.

\begin{equation}
H_{out} = \mathbf{b}(f,\mathbf{r})\mathbf{c}_{mn}^{out}
\label{eq:holo_solve}
\end{equation}
\myequations{Outgoing field solution for field separation}


Thanks to this field separation method, the speaker's SPL response can be isolated from any other components such as the effect of reflections on the room's walls, the diffraction at the edges of a system, or the acoustic shortcut between the rear and front sides of a baffle. \\
In high frequencies, where the ringing of loudspeaker is much shorter than the arrival time of the first room reflections; this method can be discarded and a time windowing applied on the Impulse Response of the system is sufficient to discriminate the outgoing and incoming waves \citep[see][sect.~3.1]{aeswb}. 



\subsection{Near Field measurements}

\subsubsection{Measurement principle}

The advantage of the Near Field compared to the Far Field measure is that the amplitude of the output is significantly higher in the Near Field, and the influence of the conditions (temperature, pressure) is minimized. \\
However, the Near-Field being much more complex than the Far Field, holographic processing is required to extrapolate the Far Field data. The complete process will be described succinctly, and can be found in details in the AES publication \textit{Holographic nearfield measurements of loudspeaker directivity} \cite{aeshs}.\\

\begin{center}
	\includegraphics[width=0.8\textwidth]{Preface/NFS_Flowchart} 
    \captionsetup{hypcap=false} 
	\captionof{figure}{NFS measurement principle flowchart} 
	\label{fig:nfsflow}
\end{center}

Figure \ref{fig:nfsflow} presents a flowchart of the measurement principle. \\ 
The first step of the scanning process is the generation of the measurement grid. The main difficulty is to avoid spatial aliasing, which is achieved by generating a non-uniform grid. Once the grid is generated, the measurement is conducted thanks to the robotics.\\
The measurement being done, the first step of the holographic process is to shift the coordinates of all measurements points to fit the expansion point $\mathbf{r}_{e}$ of the system. The expansion point is the center of the spherical waves from equation \ref{eq:sphwvexp}, and its position might change with frequency. This step is required for further processing, in order to make sure that the calculated Far Field is relevant. \\

Then, the parameters are fitted to find the optimum coefficients $C_{mn}$. This is done by estimating the deviation ${e(f,\mathbf{r}_{e}) = H_{measured}(f,\mathbf{r}_{e})-H_{model}(f,\mathbf{r}_{e})}$ between the measured and the modeled transfer function of the system, as expressed in equation \ref{eq:ParamFit}. 

\begin{equation}
\mathbf{c}(f)= \textup{arg} \; \textup{min}\left ( \sum_{\forall \mathbf{r}_{e}\in G}\left | e(f,\mathbf{r}_{e}) \right | ^{2} \right )
\label{eq:ParamFit}
\end{equation}
\myequations{Parameter fitting}

The accuracy of the holographic estimation of the sound field is then verified by estimating the Total Fitting Error ($TFE$) of the measurement. This parameter is calculated from the ratio of the truncated of the power series at a maximum order $N$ of the deviation $e(f,\mathbf{r}_{e})$ by the transfer function of the system. Evaluating the $TFE$ allows to verify the accuracy of a measurement: if it is lower than -20 dB, it is considered that the holographic model describes the Far-Field of the system with sufficient accuracy. 

\begin{equation}
TFE(f) = 10\textup{log}\left ( \frac{\sum_{\mathbf{r}_{e}\in G}\left | e(f,\mathbf{r}_{e}) \right |^{2}}{\sum_{\mathbf{r}_{e}\in G}\left | H(f,\mathbf{r}_{e}) \right |^{2}} \right )
\label{eq:TFE}
\end{equation}
\myequations{Total Fitting Error}

Another verification is the estimation of the Multi-Layer Error (MLE), given by equation \ref{eq:MLE}. This parameter allows to check the agreement between the error at the points $ \mathbf{r}_{e} $ on the outer layer and the error at the closest points $ \mathbf{r}'_{e} $ on the inner layer. 

\begin{equation}
MLE = 10log\left ( \frac{\sum_{\mathbf{r}_{e}\in G}\left | e(f,\mathbf{r}_{e})-e\textup{'}(f,\mathbf{r}\textup{'}_{e}) \right |^{2}}{\sum_{\mathbf{r}_{e}\in G} \left |H(f,\mathbf{r}_{e})  \right |^{2}-\left |H\textup{'}(f,\mathbf{r}\textup{'}_{e})  \right |^{2} } \right )
\label{eq:MLE}
\end{equation}
\myequations{Multi Layer Error}

Comparing the $ TFE $ and the $MLE$ allows to estimate the measurement's relevancy: if the errors have the same order of magnitude, it is considered that the measurement is polluted.

\subsubsection{Maximum order of expansion}
\label{sec:maxN}

The maximum order of expansion $N$ is a very important parameter to obtain accurate data. The higher it is and the more accurate the model becomes, but the longer the calculation. In that regard, a trade-off between accuracy and calculation time has to be made. The following merely aims to explain succinctly the limits on the maximum order of expansion.\\

Investigations on this parameter have already been conducted \cite{pres, aeshs, train8} and have shown that a wave expansion truncated after $N$ = 10 is capable of modeling the 3D sound pressure field of a DUT above 30 Hz with sufficient accuracy. \\
At very low frequencies the measurement is polluted by the poor Signal to Noise Ratio (SNR) between the microphone's input and the noise; and increasing the maximum order of expansion has little to no effect on the $TFE$. \\

In high frequency, increasing the number of expansion allows to reduce the $TFE$. In theory, the maximal order $N$ should be infinite but this is not possible for real applications: a computer has a limited memory size, and the engine used to interpret the diverse algorithms (in that case, SciLab) has to cope with that limit to avoid system failures.\\
Because of these physical limitations, the order of expansion can only be increased up to a certain threshold. It should be noted that the more complex the sound field, the more coefficients are required to describe it, and the lower the maximum order of expansion $N$ will be.\\


%--------------------------------------------------------------
%	BAFFLE MEASUREMENT
%--------------------------------------------------------------
\chapter{Baffle measurements}

\section{Problematic}

Performing measurements of devices mounted on a baffle requires particular attention to avoid measurement errors from the edge diffraction, the acoustical shortcut between the front and the back of the baffle, or the vibrations of the baffle itself; as shown on figure \ref{fig:baffle_problems}. \\
This section's work aims to generate a better suited measurement grid for the NFS to perform baffle measurements by modifying the "Grid Generation" block in figure \ref{fig:nfsflow}, and the critical evaluation of a baffle prototype designed to perform high accuracy baffle measurements with the NFS. 

\begin{center}
	\includegraphics[width=0.6\textwidth]{GridOpti/baffle_problems} 
    \captionsetup{hypcap=false} 
	\captionof{figure}{Baffle measurement problems} 
	\label{fig:baffle_problems}
\end{center}



\section{Grid optimization}

\subsection{Principle}
\begin{minipage}{0.6\textwidth}
For baffle measurements, the only possible scanning grid was a spherical grid, that might not be optimized for all types of drivers. A solution to this problem would be to implement different types of grids that the user could switch between in order to have a better agreement with the DUT's geometry. \\
The first grid that have been implemented is an elliptical grid. To do so, the spherical grid is generated using the existing code, and projected onto an elliptical surface of minor and major radius defined by the user, as shown in figure \ref{fig:el_grid}. \\~\\
\end{minipage}
\begin{minipage}{0.4\textwidth}
\begin{center}
	\includegraphics[width=0.9\textwidth]{GridOpti/El_Grid_Draw} 
    \captionsetup{hypcap=false} 
	\captionof{figure}{Elliptical grid} 
	\label{fig:el_grid}
\end{center}
\end{minipage}

\vspace{0.5cm}

\begin{minipage}{0.4\textwidth}
\begin{center}
	\includegraphics[width=0.8\textwidth]{GridOpti/El_Sph_Draw} 
    \captionsetup{hypcap=false} 
	\captionof{figure}{Projection of spherical grid} 
	\label{fig:proj_grid}
\end{center}
\end{minipage}
\begin{minipage}{0.6\textwidth}
Figure \ref{fig:proj_grid} shows a graphic of the projection of a circle on an ellipse. The projection to an elliptical grid have been implemented by using the following calculations: first of all, the equation of a line $y = f(x)$ intercepting y-axis in zero gives the slope $m = \frac{y_{circle}}{x_{circle}}$. Introducing in the ellipse equation and using $y_{ellipse} = m*x_{ellipse}$ leads to a solution for x, given by equation \ref{eq:ellipse}, where $a$ is the major radius and $b$ the minor radius (see figure \ref{fig:el_grid}).

\begin{equation}
\left ( \frac{x_{ellipse}}{a} \right )^{2} + \left ( \frac{y_{ellipse}}{b} \right )^{2} = 1 \; \; \;  \Leftrightarrow \; \; \;  x_{ellipse} = \sqrt{\frac{a^{2}b^{2}}{b^{2}+a^{2}m}}
\label{eq:ellipse}
\end{equation}
\myequations{Ellipse equation and solution for x}

The values of the y-coordinates of the ellipse points can then be calculated from the slope by using $y_{ellipse} = m*x_{ellipse}$. 
\end{minipage}


\subsection{Measurements}

After implementation of the elliptical grid projection in Scilab, a 10 cm woofer (see \ref{spkrlib:10cm}) have been measured using the NFS. The scan was performed using a spherical grid with a scanning radius of 20 cm, and an elliptical grid with a major radius of 20 cm and a minor radius of 5 cm. Both measurements have been done in the same conditions: equal input voltage, same number of points, and same position of the DUT. Figures \ref{fig:comp_ElSPh_SPL} and \ref{fig:comp_ElSPh_Pow} presents the magnitude and the phase of the SPL response on axis, at 10 m from the DUT. \\

\begin{minipage}{0.5\textwidth}
\begin{center}
	\includegraphics[width=.9\textwidth]{GridOpti/Mag_Compa_El_Sph}
    \captionsetup{hypcap=false} 
	\captionof{figure}{Magnitude of SPL response at 10 m} 
	\label{fig:comp_ElSPh_SPL}
\end{center}
\end{minipage}
\begin{minipage}{0.5\textwidth}
\begin{center}
	\includegraphics[width=.9\textwidth]{GridOpti/Phase_Compa_El_Sph}
    \captionsetup{hypcap=false} 
	\captionof{figure}{Phase of SPL response at 10 m} 
	\label{fig:comp_ElSPh_Pow}
\end{center}
\end{minipage}


\subsection{Discussion}

Observing the magnitude of the Far Field SPL response, it can be noted that the elliptical grid provides a much smoother curve than the spherical grid, especially in the low frequencies. This effect could be explained by the fact that the measurement points being much closer to the driver, the measure is much less sensible to systematic errors that cannot be removed by Field Separation, such as the baffle vibrating inside the scanning radius. \\
The phase of the Far Field SPL Response gives interesting results. The elliptical grid's phase presents less phase shifts, meaning that less resonances are present on this measurement, matching with the previous hypothesis of a removed systematic error due to the baffle vibrating. \\

It should be noted that the previous measurements have been performed using the field separation method (see section \ref{sec:FieldSep} of the literature review), allowing to remove the contributions from the room. In that case, going closer to the speaker allows to improve the Signal-Noise Ratio (SNR), making the measurement with the elliptical grid robuster to room modes.
As a matter of fact, the SNR can be improved by using smoothing if the measurement have been performed using the NFS and the field separation.\\

If the field separation cannot be applied on the measurement; using the elliptical grid can greatly improve the results: moving the microphone closer to the DUT makes the final results more accurate thanks to the robustness to room effects mentioned previously. \\


It can be concluded that using different measurement grid is relevant, particularly for measurements performed without field separation: a well adjusted grid have shown to give more accurate results. \\
Following this investigation, several grids could be generated, for example a half-cylindrical grid for rectangular systems mounted in baffle, such as wall speakers. Further investigations should be conducted on the influence of the measurement grid on the measurement accuracy.

\subsection{Application note for baffle measurements}

Another task was to create an application note to explain how to perform baffle measurements to Klippel's customers. This document being quite heavy, it had been chosen to not display it in this report, and is \href{https://www.dropbox.com/s/px1e5jnn0nc5rfs/AN_00_Baffle_Measurement.pdf?dl=0}{available for download by clicking \underline{here}}.

\section{Estimation of baffle influence}

\subsection{Presentation of the prototype}

\begin{minipage}{0.6\textwidth}
A baffle prototype has been designed to allow relevant measurements of devices mounted on a baffle with the NFS. The prototype is shown figure \ref{fig:baffle_proto}. More pictures are given in appendix \ref{Chap:baffle_pics}. \\

Before starting the production of this item, a critical evaluation must be performed in order to estimate the influence of the baffle on the measurement. Different analysis have been conduced, mainly to check the vibration of the baffle itself, the vibration of the wood plate used to mount the speaker, and a measurement centered on the speaker in order to have a point of comparison to identify vibration problems.
\end{minipage}
\begin{minipage}{0.4\textwidth}
\begin{center}
	\includegraphics[width=0.8\textwidth]{GridOpti/Baffle_Alone_2} 
    \captionsetup{hypcap=false} 
	\captionof{figure}{Baffle Prototype} 
	\label{fig:baffle_proto}
\end{center}
\end{minipage} 

\subsection{Measurements}

Figure \ref{fig:vib_pow} present the apparent power in the near field of the 10 cm woofer (see \ref{spkrlib:10cm}) for different measurements configurations, shown by figure \ref{fig:baffle_proto_meas}.

\vspace{0.5cm}

\begin{minipage}{0.3\textwidth}
\begin{center}
	\includegraphics[width=0.9\textwidth]{GridOpti/Baffle_meas_Points} 
    \captionsetup{hypcap=false} 
	\captionof{figure}{Measurement points} 
	\label{fig:baffle_proto_meas}
\end{center}
\end{minipage}
\begin{minipage}{0.7\textwidth}

\begin{itemize}
\item The blue curve represents the power from the speaker. This scan have been centered on the speaker, using a 5 cm radius for the scanning grid. 
\item The bright orange curve represents the contribution form the speaker and the wooden plate: this scan was centered on the speaker, with a 21 cm radius for the scanning grid.
\item The light orange curve represents the contribution from the wooden plate: this scan was centered midway between the speaker and the baffle, with a 5 cm scanning radius. 
\item  The grey curve represents the contribution from the baffle: the scan was centered on the lower right corner of the baffle, with a 5 cm scanning radius. 
\end{itemize}
\end{minipage}

\vspace{0.5cm}

Comparing the contributions from the speaker with the speaker + wood allows to estimate how the wooden plate vibration impacts the system: thanks to the field separation method, the differences between the 2 plots can only come from vibrations inside the scanning radius.\\
Comparing the variations on the speaker+wood results with the contribution of the wood shows that most of the variations matches with the peaks of the wood's contribution; whereas a few variations matches with the baffle contribution's peaks (see figure \ref{Curves:Baffle_RadPow} for matching peaks and a bigger figure). \\

Moreover, the contributions from the baffle itself aren't really problematic: if the scanning radius is correctly chosen, the field separation method will remove the baffle's contribution from the measurement. The contributions from the wooden plate are much more problematic, since the field separation method cannot remove contributions coming from inside the scanning radius, and the scanning radius cannot be smaller than the driver's. \\

Figure \ref{fig:vib_comp} compares the radiated powers in the far field (r = 10m) of the speaker (5 cm scan) and the speaker+wood (21 cm scan) measurements, evidencing the effect of the wooden plate's vibration on the measurement. At the resonance (f = 180 Hz), a strong cancellation effect is observed on the speaker+wood measurement, with a difference of almost 3 dB between the curves. This cancellation is most likely to be due to the plate vibrating in opposite phase with the speaker. \\

\begin{minipage}{0.5\textwidth} 
	\begin{center}
		\includegraphics[width=\textwidth]{GridOpti/Vib_Pow} 
    	\captionsetup{hypcap=false} 
		\captionof{figure}{Comparison of apparent power in the Near Field for different measurement points} 
	\label{fig:vib_pow}
	\end{center}
\end{minipage}
\begin{minipage}{0.5\textwidth}
	\begin{center}
		\includegraphics[width=\textwidth]{GridOpti/Vib_Pow_SPW} 
	    \captionsetup{hypcap=false} 
		\captionof{figure}{Comparison of radiated powers in the Far Field for the 5 cm and 21 cm measurements} 
		\label{fig:vib_comp}
	\end{center}
\end{minipage}
\vspace{0.1cm}

In order to have a better understanding of the wooden plate behavior, a measurement have been performed using the Klippel Scanning Vibrometer System (SCN) of the speaker mounted on the baffle. The measurement's radius have been selected to be wide enough to measure the speaker and the baffle vibrations. the results are presented in movie below \footnote{If the animation is not playing correctly, \href{https://www.dropbox.com/sh/1jlodpb3biuxmhy/AADFRz8B249n8ERu3nkwyyisa?dl=0}{please download it by clicking \underline{here}.}}, for f = 180 Hz. The plate is vibrating in opposite phase with the speaker,  which confirms the previous hypothesis of the wooden plate creating a cancellation effect on the Sound Pressure response of the driver. 


\ifdraft
\begin{center}
	\includemovie[draft,repeat,poster,label=baffle_vib,text={\small(Click to start...)}]{7cm}{5cm}{GridOpti/baffle_vib.avi}
\end{center}
\else
\begin{center}
	\includemovie[repeat,poster,label=baffle_vib,text={\small(Click to start...)}]{7cm}{5cm}{GridOpti/baffle_vib.avi}
\end{center}
\fi

\vspace{0.4cm}

Figure \ref{fig:sweep_baffle} presents the time signal measured by a microphone on the upper left corner of the baffle, recorded when exciting the baffle with a sinusoidal tone at f = 83 Hz. At this frequency, an audible rattling noise was emitted by the vibration of the nuts of the clamping system (see appendix \ref{Chap:baffle_pics} for pictures).\\
The blue curve represents the signal recorded by the microphone; and the black curve shows the residual signal, calculated by the difference between the modeled response of the system and the measured signal. \\

The residuum signal exhibits a periodicity with a period T $\approx$ 12 ms, corresponding to the frequency of the excitation. However, the frequency of the vibration is much higher than the frequency of the excitation signal: the rattling generates higher order distortions, which is why the vibrations are audible.\\
Figure \ref{fig:sweep_baffle_sono} presents the Time-Frequency Analysis (TFA) of the Transfer Function of the signal measured by a microphone at the position where rattling have been notified. The disturbances in low frequency and spreading in the time domain comes from the room influence; whereas the short time, high-frequency contributions are generated by the rattling. Although the rattling contributions have a very low power, they are clearly audible and can have a critical impact on the measurement.\\

It should be remarked that the localization of the rattling varies with the frequency of the excitation and the tightening of the clamping system; meaning that each change of setup might change the rattling behavior. 

\begin{minipage}{0.5\textwidth}
\begin{center}
	\includegraphics[width=0.9\textwidth]{GridOpti/Baffle_Rattle} 
    \captionsetup{hypcap=false} 
	\captionof{figure}{Total and residual signals of the baffle rattling} 
	\label{fig:sweep_baffle}
\end{center}
\end{minipage}
\begin{minipage}{0.5\textwidth}
\begin{center}
	\includegraphics[width=0.9\textwidth]{GridOpti/Baffle_Rattle_Sono} 
    \captionsetup{hypcap=false} 
	\captionof{figure}{Sonograph of residual signal} 
	\label{fig:sweep_baffle_sono}
\end{center}
\end{minipage}

\subsection{Discussion}

The previous investigations highlighted several problems with the baffle prototype. In order to avoid vibrations and cancellation effects, particular attention should be given to the following:
\begin{description}
\item[Wooden plate] special attention should be given to its thickness, material and damping. A solution to attenuate the vibration of the plate would be to stiffen it by attaching beams to the backside. As a general advice, MDF should be avoided because of its lightness. 
\item[Clamping system] the system clamping the plate in place could generate additional vibrations if not used correctly. The nuts supposed to hold the wooden plate in place are subject to rattling, and should be strongly tightened.
\item[Baffle] the baffle itself generates some vibration, that is believed to not be critical for the measurements since it can be removed by using the field identification. 
\end{description}

To conclude, the baffle prototype allows to generate accurate acoustic measurements of systems if the scanning radius, the plate material and the scan's geometry are selected carefully. The plate vibration and the rattling will never be totally eased off, but can be reduced by choosing a smaller scanning radius and by tightening the clamping system. \\

\begin{minipage}{0.4\textwidth}
\begin{center}
	\includegraphics[width=0.5\textwidth]{GridOpti/meas_rattling} 
    \captionsetup{hypcap=false} 
	\captionof{figure}{Plane grid for baffle diagnostics} 
	\label{fig:diag_baffle}
\end{center}
\end{minipage}
\begin{minipage}{0.6\textwidth}
The next steps of this study of the baffle is the creation of a "diagnostic tool" to identify rattling and other problems with the baffle, for example by making a coarse measurement with the NFS using a plane grid parallel to the baffle, as shown by figure \ref{fig:diag_baffle}.\\
Using holographic processing would allow to identify the localization and frequencies where the vibration is critical for the measurement by generating a "map" of the vibrations.
\end{minipage}


%--------------------------------------------------------------
%	SYMMETRY
%--------------------------------------------------------------

\chapter{Measurement time optimization}

\section{Principle}


Exploiting symmetry conditions on the complex coefficients of the spherical wave expansion $c_{n,m}^{out}(j \omega)$ (see equation \ref{eq:sphwvexp}) could reduce the measurement time significantly by reducing the number of points of the measurement as explained in the article \textit{Holographic Nearfield Measurement of Loudspeaker Directivity} \citep[][sect.~4]{aeshs}. \\

\begin{center}
	\includegraphics[width=0.7\textwidth]{Sym/flowchart} 
    \captionsetup{hypcap=false} 
	\captionof{figure}{Flowchart of the symmetry check} 
	\label{fig:sym_flow}
\end{center}

Figure \ref{fig:sym_flow} presents the flowchart of the optimization. The first step is to perform a fast measurement with a coarse grid of 150 points, approximately 20 minutes of measurement. From this measurement, it is possible to check the existence of symmetries on the DUT's geometry by using the different equations provided in the following subsection. 

\subsection{Impact of symmetries on complex coefficients of the spherical wave expansion}
\label{chap:sym}

\begin{minipage}{0.5\textwidth}
\begin{center}
	\includegraphics[width=0.8\textwidth]{Appendix/Spherical_Harmo}
    \captionsetup{hypcap=false}
    \captionof{figure}{Spherical Harmonics}
    \label{fig:Sph_Harmo}
\end{center}
\end{minipage}
\begin{minipage}{0.5\textwidth}
The solution of the three-dimensional wave equation in spherical coordinates can be described by a spherical wave expansion given by equation \ref{eq:sphwvexp}, where $c_{n,m}^{out}(j \omega)$ represents the complex coefficients weighting the spherical harmonics.\\

Figure \ref{fig:Sph_Harmo} presents a visual representation of the first few spherical harmonics, and their orders $n,m$ (\textit{all figures presented in this subsection are courtesy of W. Klippel, from the presentation \cite{lect2018}}). 
\end{minipage}
\vspace{0.4cm}

Symmetries on the DUT's geometry yield symmetries in the emitted sound field, thus on coefficients of the spherical wave expansions. This property allows to find relationship between the coefficients, allowing to reduce their number. The symmetries under study, and the derived relations between coefficients, are described bellow.    \\

\begin{minipage}{0.5\textwidth}	%Rotational Sym - text
	\textbf{Rotational symmetry} (see figure \ref{fig:rotsym}) is observed for many circular drivers and is the simplest type of symmetry as a lot of coefficients are vanishing according to the relation given by equation \ref{eq:Crotsym} \cite{aeshs}.
\end{minipage}
\begin{minipage}{0.5\textwidth}	%Rotational Sym - equation
	\begin{equation}
		C_{mn} = 0 \; \; \; \; \forall \; m \neq 0
		\label{eq:Crotsym}
	\end{equation}
	\myequations{Coefficients for rotational symmetry}
\end{minipage}

\begin{minipage}{0.5\textwidth}	% Dual plane sym - text
	\textbf{Dual plane symmetry} (see figure \ref{fig:dualpsym}) is mostly observed in 2-way closed box systems, and reduces the number of coefficients by half according to the relation given by equation \ref{eq:Cdpsym} \cite{aeshs}.\\
	$R_{m}$ represents the complex symmetry weights, expressed in equation \ref{eq:Rm} \citep{aeshs} where $\phi _{e}$ is the symmetry axis; and $m$ the suborder of the coefficient under study. \\

	If the symmetry axes $\phi _{e} \textup{and} \Theta _{e}$ are aligned with the coordinate system, equation \ref{eq:Cdpsym} can be simplified to the relation given in equation \ref{eq:Cdpsymb} \cite{aeshs}.
\end{minipage}
\begin{minipage}{0.5\textwidth}	% Dual Plane sym - equations
	\begin{equation}
		C_{mn}(f) = C_{-mn}(f)R_{m}(f) \; \; \; \;  m = 2s; \; s =1, 2, 3...
		\label{eq:Cdpsym}
	\end{equation}
	\myequations{Coefficients for dual plane symmetry}

	\begin{equation}
		R_{m}(f) = (-1)^{m+1}(\textup{sin}(m \phi _{e}(f)) + i\textup{cos}(m \phi _{e}(f)))^{2}
		\label{eq:Rm}
	\end{equation}
	\myequations{Complex symmetry parameter}

	\begin{equation}
		C_{mn}(f) = C_{-mn}(f) \; \; \; \;  m = 2s; \; s =1, 2, 3...
		\label{eq:Cdpsymb}
	\end{equation}
	\myequations{Coefficients for dual plane symmetry}
\end{minipage}


\begin{minipage}{0.5\textwidth}	%Baffle sym - text
	\textbf{Baffle symmetry} is observed  when the DUT is mounted on a baffle (see figure \ref{fig:bafflesym}), and makes half of the coefficient vanish according to the relation given by equation \ref{eq:Cbsym} \cite{aeshs}.
\end{minipage}
\begin{minipage}{0.5\textwidth}	%Baffle sym - equation
	\begin{equation}
		C_{mn} = 0 \; \; \; \; \; \; n-m \, \neq \, 2s; \: \: s \in \mathbb{Z}
		\label{eq:Cbsym}
	\end{equation}
	\myequations{Coefficients for baffle symmetry}
\end{minipage}
\vspace{0.2cm}

\begin{minipage}{0.5\textwidth}	%Single plane - text
	\textbf{Single plane symmetry} is the most common in loudspeaker systems using 2 or more drivers (see figure \ref{fig:singplsym}). Only the coefficients on either the left or right side have to be estimated; and the remaining coefficients are calulated by using equation \ref{eq:spsym} \cite{aeshs}. 
\end{minipage}
\begin{minipage}{0.5\textwidth}	% Single Plane sym - equation
	\begin{equation}
		C_{mn} = C_{-mn}(f)R_{m}(f) \; \; \; \textup{with} \; \; m\geq 0; \; \; \; 0\leq n\leq N
		\label{eq:spsym}
	\end{equation}
	\myequations{Coefficients for single plane symmetry}
\end{minipage}
\vspace{0.1cm}

The Single Plane Symmetry can occur between the left and right side of the DUT, as presented in figure \ref{fig:singplsym}; but also between the top and bottom of the DUT, shifting the symmetry plane by 90°. It should be noted that depending of the type of symmetry (left/right or top/bottom), $R_{m}$ changes, modifying the relation between the coefficients. 

\begin{minipage}{0.248\textwidth}	% Rot - img
	\begin{center}
		\includegraphics[width=0.9\textwidth, height=5cm]{Appendix/Rot_Sym}
   	 	\captionsetup{hypcap=false}
   	 	\captionof{figure}{Rotational}
    	\label{fig:rotsym}
	\end{center}
\end{minipage}
\begin{minipage}{0.248\textwidth}	%DPS - Img
	\begin{center}
		\includegraphics[width=0.9\textwidth, height=5cm]{Appendix/Dual_Plane_Sym}
    	\captionsetup{hypcap=false}
    	\captionof{figure}{Dual Plane}
    	\label{fig:dualpsym}
	\end{center}
\end{minipage}
\begin{minipage}{0.248\textwidth}	%Baffle - img
	\begin{center}
		\includegraphics[width=0.9\textwidth, height=5cm]{Appendix/Baffle_Sym}
    	\captionsetup{hypcap=false}
    	\captionof{figure}{Baffle}
    	\label{fig:bafflesym}
	\end{center}
\end{minipage}
\begin{minipage}{0.248\textwidth} %SPS - img
	\begin{center}
		\includegraphics[width=0.85\textwidth, height=5cm]{Appendix/Single_Plane_Sym}
    	\captionsetup{hypcap=false}
   	 	\captionof{figure}{Single Plane}
    	\label{fig:singplsym}
	\end{center}
\end{minipage}

\subsection{Implementation}
\label{sec:sym_imple}

Once a measurement have been performed, the complex outgoing coefficients are identified and expressed in a matrix form shown in equation \ref{eq:matrix_cmn}. Each row represent a $m,n$ order and suborder, and each column a frequency.

%-------------------------- Coeffs matrix
\begin{equation}
\label{eq:matrix_cmn}
C_{mn} = \qquad \bordermatrix{
						      & \tikzmark{harrowleft}  &     &     & \tikzmark{harrowright}\cr
\tikzmark{varrowtop} & C_{0,0}(f_{1})           & ... & ... & C_{0,0}(f_{n})   \cr 
        					  & C_{-1,1}(f_{1})         & ... & ... & C_{-1,1}(f_{n}) \cr
         					  & C_{0,1}(f_{1})          & ... & ... & C_{0,1}(f_{n})   \cr
         					  & C_{1,1}(f_{1})          & ... & ... & C_{1,1}(f_{n})   \cr 
         					  & ...                            & ... & ... & ...                    \cr
\tikzmark{varrowbottom} & C_{m,n}(f_{1})     & ... & ... & C_{m,n}(f_{n}) \cr }
\end{equation}
\myequations{Matrix representation of coefficients}
\tikz[overlay,remember picture] {
  \draw[->] ([yshift=1.1ex,xshift=1ex]harrowleft) -- ([yshift=1.1ex]harrowright)
            node[midway,above] {\scriptsize $f$};
  \draw[->] ([yshift=1ex,xshift=-1ex]varrowtop) -- ([xshift=-1ex]varrowbottom)
            node[near end,left] {\scriptsize $m,n$};
}

From the complete matrix of coefficients and the relationships expressed in section \ref{chap:sym}; it is possible to derive matrices containing only the coefficients satisfying each symmetry condition.\\
The algorithm of the symmetry verification is given in appendix \ref{Chap:imple_sym}; and the following describes its principle. \\
The first step is to initialize the matrices that will contain the coefficients by filling them with zeros and making them the same size than the complete matrix of coefficients.\\

For the rotational symmetry, the indexes of the rows containing the coefficients of order $m = 0$ are calculated, and the symmetry matrix is generated by replacing the zeroes in the previously initialized matrix by the coefficients at the corresponding indexes. the resulting matrix is given by equation \ref{eq:matrix_cmn_ROT}.

\begin{equation}
\label{eq:matrix_cmn_ROT}
C_{mn_{rotational}}  = \begin{pmatrix}
							 C_{0,0}(f_{1})           & ... & ... & C_{0,0}(f_{n})   \\ 
        					   0        & ... & ... & 0 \\
         					  C_{0,1}(f_{1})          & ... & ... & C_{0,1}(f_{n})   \\
         					   0         & ... & ... & 0   \\
							  ...                          & ... & ... & ...                  \\ \end{pmatrix}
\end{equation}
\myequations{Matrix of rotational symmetry coefficients}

For the single plane symmetry, the indexes of the coefficients on the left side satisfying $m \geq 0$ and $0 \leq n \leq N$ are calculated and the left-side symmetry matrix is filled using the same method as for the rotational symmetry. \\
The indexes corresponding to the coefficients on the right side satisfying $m_{right} = -m_{left}$ are then calculated; and the coefficients on the left are mirrored to their corresponding indexes in the right-side matrix, as shown by equation \ref{eq:matrix_SPS}.

\begin{equation}
C_{mn_{Left}} = \begin{pmatrix}
    C_{0,0}(f_{1})    & ... & ... & C_{0,0}(f_{n})   \cr 
    C_{-1,1}(f_{1})   & ... & ... & C_{-1,1}(f_{n}) \tikzmark{aarrowtop}\cr
     C_{0,1}(f_{1})    & ... & ... & C_{0,1}(f_{n})   \cr
     0       & ... & ... & 0   \cr
     C_{-2,2}(f_{1})    & ... & ... & C_{-2,2}(f_{n})  \tikzmark{barrowtop} \cr
     C_{-1,2}(f_{1})    & ... & ... & C_{-1,2}(f_{n})  \tikzmark{carrowtop} \cr 
    ...               & ... & ... & ...               \cr \end{pmatrix} \quad
C_{mn_{Right}} = \begin{pmatrix}
     0    & ... & ... & 0   \cr 
     0   & ... & ... & 0 \cr
     0    & ... & ... & 0   \cr
 \tikzmark{aarrowbottom}    C_{1,1}(f_{1})    & ... & ... & C_{1,1}(f_{n})   \cr
    0    & ... & ... & 0   \cr
    0    & ... & ... & 0  \cr 
    0    & ... & ... & 0   \cr 
  \tikzmark{carrowbottom}    C_{1,2}(f_{1})    & ... & ... & C_{1,2}(f_{n})  \cr 
  \tikzmark{barrowbottom}  C_{2,2}(f_{1})    & ... & ... & C_{2,2}(f_{n})\cr 
    ...               & ... & ... & ...               \cr \end{pmatrix}
\label{eq:matrix_SPS}
\end{equation}
\myequations{Matrix of single plane symmetry coefficients}

\tikz[overlay,remember picture] {
  \draw[dashed,->] ([xshift=1.5ex]aarrowtop) -- ([xshift=-1.5ex,yshift=0.5ex]aarrowbottom);
  \draw[dashed,->] ([xshift=1.5ex]barrowtop) -- ([xshift=-1.5ex,yshift=0.5ex]barrowbottom);	
  \draw[dashed,->] ([xshift=1.5ex]carrowtop) -- ([xshift=-1.5ex,yshift=0.5ex]carrowbottom);
}

The $R_{m}$ weights matrix is calculated for each $mn$ orders and each frequency using equation \ref{eq:Rm}. For the left/right symmetry the angle $\phi_{e}$ is $\frac{\pi}{2}$ and for the top/bottom symmetry, $\phi_{e} = 0$; changing the values of the weights. The right-side matrix is then weighted by $R_{m}$ by multiplying the matrices: $C_{mn_{Right}} = R_{m} \times C_{mn_{Right}}$. \\
To avoid overlapping and ensure that the single plane symmetry coefficient matrix is correct, the coefficients of order $m = 0$ are only filled once in the left-side matrix. The coefficient matrix is then computed by adding the right-side and left-side matrices $C_{mn_{SinglePlane}} = C_{mn_{Left}} + C_{mn_{Right}}$. \\


For the baffle and the dual plane symmetries, the same principle as for the single plane symmetry is applied to retrieve the coefficient matrix. Of course, the indexes of the relevant coefficients will change with the symmetry under study. There is no need the multiply the right-side matrix by the weights matrix $R_{m}$ for the dual plane and baffle symmetries, as seen in subsection \ref{chap:sym}. \\

From the coefficients matrices the radiated power is calculated by using equation \ref{eq:power}.
\begin{equation}
	\Pi(f) = \frac{1}{2\rho _{0}c k^{2}} \sum \mid C_{mn}(f)\mid ^{2}
	\label{eq:power}
\end{equation}
\myequations{Radiated power from complex coefficients}

Now that the coefficient matrix is known, the rest matrix is calculated by subtracting the symmetry coefficients matrix to the total coefficients matrix: $C_{rest} = C_{total} - C_{symmetry}$. The rest matrix allows to characterize how much the system is fulfilling the symmetry conditions by comparing the power radiated by the symmetry coefficients and the power radiated by the rest: if the rest's power is higher than the symmetry's power then the system is not fulfilling the condition.\\

Another method to assess the symmetry of a system is calculating the ratio of the power radiated by the symmetry and the total power. This parameter is called the symmetry factor, and is given by equation \ref{eq:symfqct}.

\begin{equation}
	S = \frac{\Pi_{sym}}{\Pi_{total}}
	\label{eq:symfqct}
\end{equation}
\myequations{Symmetry factor}

The symmetry factor can be expressed in dB, and allows to quantify the variations between the total radiated power and the power radiated by the symmetry condition of interest. The symmetry presenting the less differences is called the "dominant symmetry". \\

Once the dominant symmetry is identified, the user is informed that a faster scan can be performed. For example if the Dual Plane Symmetry is found to be dominant, it is possible to scan only a fourth of the device and to reconstruct the emitted sound field by mirroring the results to the remaining unmeasured quadrants.\\

Another use of the symmetry can be in post-processing: after finding a symmetry for a device, it is possible to speed the processing by removing the irrelevant coefficients. For the Dual Plane Symmetry example, all coefficients  $C_{mn} \;  \textup{for} \;  m = 2s; \; s =1, 2, 3...$ are null, thus reducing the number of coefficients by 70\%.\\
This reduction of the number of coefficients can be applied to all of the symmetry conditions, and can speed up the post-processing significantly. Having less coefficients would also allow to increase the order of expansion $N$: less memory will be required as explained in section \ref{sec:maxN}; leading to a more accurate model of the 3D sound field emitted by the DUT and a higher frequency resolution.

\newpage

\section{Measurements}

After implementation in SciLab (the code is given in appendix \ref{Chap:imple_sym}) , the validity of the symmetry assumptions must be verified. To do so, different measurements have been performed of several speakers presenting interesting geometries. A library of these speakers is given in appendix \ref{chap:spk_lib}. 


\subsection{10 cm woofer}

The DUT is a circular, 10 cm woofer (see appendix \ref{spkrlib:10cm}), mounted on the baffle. \\

\begin{minipage}{0.5\textwidth}
\begin{center}
	\includegraphics[width=\textwidth]{Sym/Rad_Pow_10cmWoofer} 
    \captionsetup{hypcap=false} 
	\captionof{figure}{Radiated Powers of 10cm woofer} 
	\label{fig:rad_pow_10cm}
\end{center}
\end{minipage}
\begin{minipage}{0.5\textwidth}
\begin{center}
	\includegraphics[width=\textwidth]{Sym/Sym_Fact_10cmWoofer} 
    \captionsetup{hypcap=false} 
	\captionof{figure}{Symmetry Factors of 10cm woofer} 
	\label{fig:sym_fact_10cm}
\end{center}
\end{minipage}
\vspace{0.1cm}

Figure \ref{fig:rad_pow_10cm} presents the calculated radiated powers for different symmetry conditions. The dashed curves represents the "rest", which is the power radiated by coefficients not satisfying the symmetry condition described in section \ref{chap:sym}, in the example of a rotational symmetry the rest is constituted of all coefficients $C_{mn} \neq 0 \;  \textup{for m}  \neq 0$. For a better reading, all the radiated power curves are provided in separate graphs in appendix \ref{Curves:10cm}.\\

Figure \ref{fig:sym_fact_10cm} presents the symmetry factors for different types of symmetry, expressed in dB. A 0 dB value means that the power radiated by considering only the coefficients satisfying the symmetry condition and the total radiated power present no differences. Observing the symmetry factor for the baffle symmetry shows that this value is 0 dB over the whole frequency band: this result is matching the expectations since the measurement have been done with the speaker mounted on a baffle. \\

Comparing the radiated power in figure \ref{fig:rad_pow_10cm} and appendix \ref{Curves:10cm} for the different symmetries gives interesting results: in the low frequency domain, the total radiated power and the powers radiated by the different symmetry conditions presents no differences, no matter the symmetry type. In low frequency, the driver behaves as a monopole and have an omnidirectional directivity pattern; fulfilling all the symmetry conditions. \\
However, in high frequencies, disruptions appears and the rest increases significantly: the higher the frequency, and the more interferences are produced by the speaker. This can be observed on the contour plot of the woofer's directivity, presented in appendix \ref{fig:contour_10cm}. \\

Comparing the symmetry factors shows that the most present symmetries are the Single Plane symmetries. This result may be surprising and one could expect the rotational symmetry to be dominant; but as seen on the contour plot in high frequencies the speaker isn't a monopole: the cone breakups spoil the rotational symmetry. 

\newpage

\subsection{Oval speaker}
The DUT is an oval speaker (see appendix \ref{spkrlib:oval}), mounted on the baffle. The particular geometry of this speaker gives hints about the expected results: the rotational symmetry should be the lowest one, and the planar symmetries should be dominant (single plane or dual plane symmetries).  Figures \ref{fig:rad_pow_Oval} and \ref{fig:sym_fact_Oval} presents the radiated powers and the symmetry factors respectively. The radiated powers are presented in separated graphs in appendix \ref{Curves:oval}.\\

\begin{minipage}{0.5\textwidth}
\begin{center}
	\includegraphics[width=\textwidth]{Sym/Rad_Pow_OvalSpkr} 
    \captionsetup{hypcap=false} 
	\captionof{figure}{Radiated Powers of oval speaker} 
	\label{fig:rad_pow_Oval}
\end{center}
\end{minipage}
\begin{minipage}{0.5\textwidth}
\begin{center}
	\includegraphics[width=\textwidth]{Sym/Sym_Fact_OvalSpkr} 
    \captionsetup{hypcap=false} 
	\captionof{figure}{Symmetry Factors of oval speaker} 
	\label{fig:sym_fact_Oval}
\end{center}
\end{minipage}
\vspace{0.1cm}

Observing the radiated powers for the different symmetries, it is found that in the 20 Hz - 1.5 kHz range there is no differences between the total radiated power and the symmetries radiated power. Similarly to the 10 cm woofer, the drivers behaves as a monopole in this range. This is observed on the contour plot provided in figure \ref{Curves:oval_contour}. \\
After 1.5 kHz, differences appear for the Dual Plane, Single Plane and Rotational symmetries. Characterizing the speaker's behavior in higher frequencies requires to look at the symmetry factors. However, it can already be observed that the power radiated by rotational symmetry coefficients is below all other radiated powers after 1.5 kHz, meaning that the rotational symmetry condition is not satisfied. \\

Observing the symmetry factors shows that the rotational symmetry is below all other factors, matching with the expected behavior of the driver. The baffle symmetry factor is 0 dB over the whole frequency range, as expected for a measure performed with a driver mounted on baffle.\\

\begin{minipage}{0.5\textwidth}
Interestingly, the top/bottom single plane symmetry factor is higher than the other factors from 5 kHz. This could be explained by the speaker's geometry: the larger dimension makes it more prone to rocking mode, spoiling the left/right single plane symmetry. At 9400 Hz there is a peak on the symmetry factors (and a dip on the radiated powers), probably due to a membrane mode.\\

Figure \ref{fig:phase_Oval} presents the phase balloon of the oval speaker at 9400 Hz. It can be observed that the left and right sides of the membrane vibrate in opposite phase, validating the hypothesis of the left/right symmetry condition not being satisfied because of the rocking modes of the driver's membrane. 
\end{minipage}
\begin{minipage}{0.5\textwidth}
\begin{center}
	\includegraphics[width=0.8\textwidth]{Sym/phase_oval} 
    \captionsetup{hypcap=false} 
	\captionof{figure}{Phase balloon at f = 9400 Hz of oval speaker} 
	\label{fig:phase_Oval}
\end{center}
\end{minipage}

\newpage

\subsection{2-way "diagonal" speaker}

The DUT is a "diagonal" 2-way speaker, and the scan is centered on the tweeter (see appendix \ref{spkrlib:BnO}). This speaker's asymmetrical geometry is very interesting for the symmetry measurement. The device also have an integrated crossover. Figures \ref{fig:rad_pow_BnO} and \ref{fig:sym_fact_BnO} presents the radiated powers and the symmetry factors respectively. The radiated powers are plotted in separated graphs in appendix \ref{Curves:2way}. \\

\begin{minipage}{0.5\textwidth}
\begin{center}
	\includegraphics[width=\textwidth]{Sym/Rad_Pow_BnO} 
    \captionsetup{hypcap=false} 
	\captionof{figure}{Radiated Powers of diagonal speaker} 
	\label{fig:rad_pow_BnO}
\end{center}
\end{minipage}
\begin{minipage}{0.5\textwidth}
\begin{center}
	\includegraphics[width=\textwidth]{Sym/Sym_Fact_BnO} 
    \captionsetup{hypcap=false} 
	\captionof{figure}{Symmetry Factors of diagonal speaker} 
	\label{fig:sym_fact_BnO}
\end{center}
\end{minipage}

\vspace{0.4cm}

Observing the radiated powers shows that the system have an omnidirectional directivity pattern until 175 Hz, also visible on the contour plot given in figure \ref{Curves:2way_contour}. Due to its geometry and 2 drivers, the system less susceptible to present symmetries than a single speaker mounted on a baffle seen in the previous examples.\\
The crossover frequency range can be observed clearly on both result result curves: the symmetry factors take a dip, and the rest powers are higher than the symmetrical powers. \\

Observing the Symmetry Factors shows that there is no baffle symmetry for this measurement. Although this result might seem quite trivial since the DUT was not mounted on a baffle, it is good indication that the symmetry estimation works correctly. \\
Around crossover frequency range (800 Hz - 3.5 kHz), both drivers of the DUT are working, and due to their diagonal positioning the system cannot fulfill any of the symmetry conditions. The highly asymmetric behavior of the system is very visible on the contour plot given in figure \ref{Curves:2way_contour}. \\
In high frequency when the twitter alone is emitting, the symmetry factors of the rotational and planar symmetries increase again, meaning that the sound field emitted by the tweeter fulfill these conditions. 

\section{Discussion}

The previous results are promising for the investigation on symmetry conditions to ease off the measurement or processing time. The calculated symmetry factors match with the expected results for all measured systems, and can be implemented in the NFS scripts to be exploited for reducing the number of measurements points. \\

The symmetry verification algorithm described in section \ref{sec:sym_imple} and given in appendix \ref{Chap:imple_sym} is ready to be integrated to the NFS field identification module, and the next step of this work will be to generate the grids for each symmetry (for example a line scan for the rotational symmetry). The symmetries can also be combined, for example it is possible to generate a quarter of an half-sphere to measure a device mounted on baffle that presents a Dual Plane symmetry. 

%--------------------------------------------------------------
%	ROOM CORRECTION
%--------------------------------------------------------------
\chapter{Room Compensation}

Being able to generate room compensation curves is great for the everyday measurement of loudspeakers since it allows to compensate the effect of any environment and to create quasi-perfect measurements. According to the AES standard \cite{aesstandart}, "\textit{accurate and repeatable loudspeaker driver measurements require that environmental and boundary influence have minimal impact on the measurement}". Such measurements are usually done in anechoic rooms, and accurate results are difficult to achieve for companies not having access to such a room or for a hobbyist.

\section{Principle}

By using post-processing, it is possible to simulate free field conditions and to compensate the influence of the room on the measurement. Several methods are available to generate this compensated curves, from time windowing to the generation of compensation functions \citep[see][]{aeswb}. Describing all of these methods is beyond the scope of this report, and this section is focused on the compensation by a reference curve method. \\

\begin{minipage}{0.331\textwidth}
\begin{center}
	\includegraphics[width=0.95\textwidth]{RoomComp/Sonograph_Meas} 
    \captionsetup{hypcap=false} 
	\captionof{figure}{Measurement result} 
	\label{fig:comp_meas}
\end{center}
\end{minipage}
\begin{minipage}{0.331\textwidth}
\begin{center}
	\includegraphics[width=0.95\textwidth]{RoomComp/Sonograph_MagFilt} 
    \captionsetup{hypcap=false} 
	\captionof{figure}{Filter on magnitude} 
	\label{fig:comp_mag}
\end{center}
\end{minipage}
\begin{minipage}{0.331\textwidth}
\begin{center}
	\includegraphics[width=0.95\textwidth]{RoomComp/Sonograph_RoomCor} 
    \captionsetup{hypcap=false} 
	\captionof{figure}{Filter with LFR} 
	\label{fig:comp_RoomCor}
\end{center}
\end{minipage}

\vspace{0.1cm}

Figures \ref{fig:comp_meas}, \ref{fig:comp_mag} and \ref{fig:comp_RoomCor} presents the Time-Frequency Analysis (TFA) of the measurement of the 10 cm woofer (see \ref{spkrlib:10cm}) Transfer Function (TRF) in free air. Different methods have been applied to simulate free field conditions and compensate for the room influence:

\begin{description}
\item[Figure \ref{fig:comp_meas}] presents the measurements without any compensation applied. The decay of the room have a large impact on the measurement, and makes the speaker's behavior difficult to discriminate from the room's artifacts. In low frequencies, the room modes are very remarkable and have a critical impact on the measure because of how they spread over the whole measurement time. In high frequencies, the modes are less critical and could be eased-off by smoothing the TRF.

\item[Figure \ref{fig:comp_mag}] a pre-filter have been applied on the magnitude of the TRF, using the method shown in figure \ref{fig:flow_postfilt}, where the filter $H_{c}$ is the difference between a reference measurement done with the NFS and the TRF measurement. This method is the most straight-forward to implement, but the filter $H_{c}$ has a zero phase and is only applied on the magnitude. The lack of phase information generates the acausality and other artifacts observed on the TFA.

\begin{center}
	\includegraphics[width=0.4\textwidth]{RoomComp/flowchart_prefilter} 
    \captionsetup{hypcap=false} 
	\captionof{figure}{Pre-filter flowchart} 
	\label{fig:flow_postfilt}
\end{center}

\item[Figure \ref{fig:comp_RoomCor}] the TRF have been post-filtered by a Room Compensation reference curve calculated using the Low-Frequency Reference method described in \citep[][sect.~4]{aeswb} and shown in figure \ref{fig:flow_LFR}. This method provides the best result: by using a reference curve provided by a measurement of the woofer using the NFS, the phase response of the room can be isolated from the measurement, removing the ringing artifacts visible on the TFA. 
\end{description}

\begin{center}
	\includegraphics[width=0.4\textwidth]{RoomComp/flowchart_LFR} 
    \captionsetup{hypcap=false} 
	\captionof{figure}{Low-Frequency Reference flowchart} 
	\label{fig:flow_LFR}
\end{center}

\vspace{0.2cm}

The Low-Frequency Reference method gives accurate results, but requires a good reference curve to be relevant: if the reference curve isn't accurate, the compensation will give erroneous results. \\
In the following, different methods of obtaining the reference curve and different exploitation of the Low-Frequency Reference method will be tested. 

\section{Compensation by a Near Field measurement}

\begin{minipage}{0.4\textwidth}
\begin{center}
	\includegraphics[width=0.7\textwidth]{RoomComp/NFFF_Setup} 
    \captionsetup{hypcap=false} 
	\captionof{figure}{Measurement setup} 
	\label{fig:SetupNFFF}
\end{center}
\end{minipage}
\begin{minipage}{0.6\textwidth}
A very simple and straightforward approach to generate the Reference Curve is to exploit a measure performed in the near field of the DUT. Figure \ref{fig:SetupNFFF} presents the measurement set-up: the reference point is located at 5 cm on-axis of the speaker; and the measurements are taken in the far field at 0.5 m and for different angles. The DUT is the 10 cm woofer (see appendix \ref{spkrlib:10cm}). \\

Figure \ref{fig:rawNFFF} presents the measurement results without compensation; and figure \ref{fig:resNFFF} presents the different far field measurement when compensated using the near field reference curve and the Low-Frequency Reference method described in \citep[][sect.~4]{aeswb}. 
\end{minipage}

\vspace{0.4cm}

\begin{minipage}{0.5\textwidth}
\begin{center}
	\includegraphics[width=0.8\textwidth]{RoomComp/NF_FF_compa} 
    \captionsetup{hypcap=false} 
	\captionof{figure}{TRF at different measurement points} 
	\label{fig:rawNFFF}
\end{center}
\end{minipage}
\begin{minipage}{0.5\textwidth}
\begin{center}
	\includegraphics[width=0.8\textwidth]{RoomComp/NF_FF_result} 
    \captionsetup{hypcap=false} 
	\captionof{figure}{Results with compensation} 
	\label{fig:resNFFF}
\end{center}
\end{minipage}
\vspace{0.2cm}

The results shows a good match before the cross-frequency range (shown by the vertical yellow bars in figure \ref{fig:resNFFF}); where the measure is compensated; but have poor results in higher frequencies. Measurements in the near field of speakers are polluted by the interferences effects, and even if the compensation was applied to the whole frequency range, these near field effects couldn't be eased off in high frequencies.\\
Furthermore, if the DUT has several diaphragms (for example a 2-way system), this method isn't applicable: the drivers interfering with each other would make the reference curve inaccurate, giving erroneous results. 

\section{Compensation by a NFS result}

This method uses the same principle as the compensation by a nearfield measurement, but the reference curve is now generated from a Near Field Scanner measurement. The measurement curves are the same as the previous section (far field on-axis, far field 10° off-axis and far field 20° off-axis). The results are given in figure \ref{fig:NFSc}.\\

\begin{center}
	\includegraphics[width=0.4\textwidth]{RoomComp/NFS_compa} 
    \captionsetup{hypcap=false} 
	\captionof{figure}{Results with NFS compensation} 
	\label{fig:NFSc}
\end{center}

In the low frequency range, there is a good agreement between the results; but from 1.5 kHz differences start to appear. These differences are due to the reference curve: to generate the result, the reference curve have been calculated on-axis at 0.5 m; or to compensate for an off-axis measurement the reference curve should be calculated off-axis. Although it can be fixed easily, it underlines another problematic with the compensation of free-field measurement: the reference curve must be recalculated (or remeasured) for each change in the measurement configuration.

\section{D. B. Keele method for Far Field extrapolation}
\label{sec:dBKeele}

D. B. Keele presented a simplified method for performing the extrapolation of the far field pressure of a device by measuring in the near field. The theory is given in the article \textit{Low-Frequency Loudspeaker Assessment by Nearfield Sound-Pressure Measurement} \citep[see][]{dkeele}. The near - far pressure relation is given in equation \ref{eq:nfpressure}.

\begin{equation}
p_{N} = \frac{2r}{a} \cdot p_{F}
\label{eq:nfpressure}
\end{equation}
\myequations{Near-Far field pressure relation}

Where $p_{N}$ and $p_{F}$ are respectively the near and far field pressures, $r$ is the distance between the center of the driver's membrane and the measuring point, and $a$ the radius of the driver's membrane.\\
Figure \ref{fig:dbk} presents the results of the far field extrapolation applied to the free air near field measurement presented in figure \ref{fig:rawNFFF}, compared with the measure in far field and the compensated far field result. \\

\begin{center}
	\includegraphics[width=0.4\textwidth]{RoomComp/dbkeele} 
    \captionsetup{hypcap=false} 
	\captionof{figure}{D.B. Keele Far Field extrapolation} 
	\label{fig:dbk}
\end{center}

In the low frequency domain, the method gives accurate results and there is an agreement between all of the curves. Strong discrepancies appear around 2 kHz, where the interferences effects starts to impact the near field measurement. In his paper, D. B. Keele has found the pressure to be constant up to the frequency where $a / \lambda$ = 0.26; with $\lambda$ the wavelength; meaning that the measurement in the near field is only accurate up to a certain frequency, depending on the DUT's radius \citep[see][section Nearfield Pressure]{dkeele}. \\

Unlike the compensation by a near field measurement, the Far Field extrapolation is only valid on axis of the DUT.\\
The extrapolation can be applied to multiple diaphragms systems providing the near field responses to be measured individually and summed in the ratio of the square root of their diameter. The sums needs to be complex to ensure the phase shift associated with each diaphragm is considered and give accurate results. \citep[][]{dkeele}

\section{Tetrahedral box}
Although the previous methods are effective in their respective frequency range, they require the reference curve to be remeasured for each environment or change of the DUT position, as the room effects will vary with the positioning of the loudspeaker. A solution to this problematic is to measure in test boxes, such as the tetrahedral test box created by Hill Acoustics \citep[see][]{tetbox}: the boxes provide a shielding against the outside environment, allowing a reproductibility of results. 

\subsection{Influence of microphone position}

Before experimenting with the Room Compensation functions for the box, the accuracy of the measurements must be ensured.\\ 
Investigating the influence of the microphone position allows to deduce how critical this parameter is, and how to ensure a correct microphone positioning. Several measurements of the 10 cm woofer (see appendix \ref{spkrlib:10cm}) have been performed for different microphone - speaker distances $D_{ms}$. A schematic of the measurement is given in figure \ref{fig:micpos_schema}.  \\

Figure \ref{fig:MicPosTRF} shows the transfer functions obtained for different distances $D_{ms}$ (see figure \ref{fig:micpos_TRF_big} for a bigger version). In low frequencies up to 150 Hz, the transfer functions are similar. After 150 Hz strong differences appear between the transfer functions. The larger the distance $D_{ms}$ is and the bigger the box influences the measurement, generating a dip on the response at around 1 kHz. This phenomenon is believed to be the first box mode. \\

\begin{minipage}{0.35\textwidth}
\begin{center}
	\includegraphics[width=0.9\textwidth]{RoomComp/Mic_Pos_Schm} 
    \captionsetup{hypcap=false} 
	\captionof{figure}{Measurement setup} 
	\label{fig:micpos_schema}
\end{center}
\end{minipage}
\begin{minipage}{0.6\textwidth}
\begin{center}
	\includegraphics[width=1\textwidth]{RoomComp/MicPos_TRF} 
    \captionsetup{hypcap=false} 
	\captionof{figure}{Transfer Functions for different distances $D_{ms}$} 
	\label{fig:MicPosTRF}
\end{center}
\end{minipage}
\vspace{0.2cm} 

From the previous observations it can be deduced that there is a critical range for the microphone positioning to avoid the influence of the mode and ensure an accurate measure. Figure \ref{fig:MicPosDiffNFS} shows the differences between the results from the NFS, extrapolated at their respective distances; and the measured transfer functions. This set of curves is the Room Compensation functions, and the NFS results are the references. In the 20 Hz - 150 Hz frequency range a "bass boost" is observed, created by the box's air load. This effect can be compensated and the matter is discussed in section \ref{sec:load}.\\

Figure \ref{fig:MicPosDiffRC} shows the difference between the Room Compensation function generated at $D_{ms}$ = 5cm and the Room Compensation functions at $D_{ms}$ = 6 cm, 7 cm, 8 cm and 15 cm. If the microphone position had no influence on the measurement, there should theoretically be no differences between the Room Compensation functions. \\

\begin{minipage}{0.5\textwidth}
\begin{center}
	\includegraphics[width=0.9\textwidth]{RoomComp/MicPos_Diff_NFS_TRF} 
    \captionsetup{hypcap=false} 
	\captionof{figure}{Room Compensation functions} 
	\label{fig:MicPosDiffNFS}
\end{center}
\end{minipage}
\begin{minipage}{0.5\textwidth}
\begin{center}
	\includegraphics[width=0.9\textwidth]{RoomComp/MicPos_Diff_RefMEas} 
    \captionsetup{hypcap=false} 
	\captionof{figure}{Difference between Room Compensation functions} 
	\label{fig:MicPosDiffRC}
\end{center}
\end{minipage}
\vspace{0.1cm}

As a matter of fact, obtaining the exact same Room Compensation function for 2 measurement performed in the same conditions and for the same $D_{ms}$ is unlikely. \\
To assess the accuracy of the measurement, a difference level of $\pm$ 5 dB between the Room Compensation functions have been arbitrary chosen. The distances satisfying these conditions are the measurements at $D_{ms}$ = 6 cm and $D_{ms}$ = 7 cm. At f = 6500 Hz and f = 10 kHz, the Compensation function for $D_{ms}$ = 7 cm exhibits larger differences with the Compensation function for $D_{ms}$ = 5 cm, and measuring at a larger distance should be avoided. 

\subsection{Influence of the load}
\label{sec:load}

The tetrahedral test box being a closed box system, it has an impact on the speaker's behavior. A closed box can be approximated as an additional  stiffness on the speaker's equivalent network, due to the springiness of the air volume trapped in the box. This effect is mostly present in low frequency, as shown in figure \ref{fig:Load_Compa}: the blue curve is the measure in the box and the orange curve the result of the NFS of the 10 cm woofer (see appendix \ref{spkrlib:10cm}). In the 20 Hz - 150 Hz range the box creates a "bass boost" effect. \\
Figure \ref{fig:kms_comp} presents the measured mechanical stiffness $K_{ms}$ in free air and in the box. This measurements evidences how the box's load influences the stiffness of the membrane: providing the driver is still in its linear range, the $K_{ms}$ curve is merely shifted. \\ 

\begin{minipage}{0.5\textwidth}
\begin{center}
	\includegraphics[width=0.7\textwidth]{RoomComp/Compa_NFS_BAffle} 
    \captionsetup{hypcap=false} 
	\captionof{figure}{Comparison between the TRF of a measure in baffle and a measure in the box.} 
	\label{fig:Load_Compa}
\end{center}
\end{minipage}
\begin{minipage}{0.5\textwidth}
\begin{center}
	\includegraphics[width=0.7\textwidth]{RoomComp/diff_kms} 
    \captionsetup{hypcap=false} 
	\captionof{figure}{Comparison between the $K_{ms}$ in free air and in the box} 
	\label{fig:kms_comp}
\end{center}
\end{minipage}

\vspace{0.2cm}

Assuming that the driver is in the linear range, the displacement transfer function of a driver in free air is given by equation \ref{eq:TRF_disp}, where $ e_{g}$ is the input voltage, $C_{ms}$ the compliance, $Bl$ the force factor, $\omega_{s}$ the resonance frequency of the system, and $Q_{ts}$ the total quality factor of the system.

\begin{equation}
x(\omega ) = e_{g}\cdot \frac{C_{ms}Bl}{R_{e}}\cdot \frac{1}{(j\omega /\omega _{s})^{2}+(1/Q_{ts})(j\omega /\omega _{s})+1}
\label{eq:TRF_disp}
\end{equation}
\myequations{Displacement transfer function in free air}

Figure \ref{fig:eq_circ} represents the simplified equivalent circuit of the driver in the box. The effect of the box is represented by the compliance $C_{box}$. $e_{g}$ represents the input voltage, $R_{e}$ the electrical resistance; $R_{ms}$, $C_{ms}$ and $M_{ms}$ the mechanical parameters of the driver and $Bl$ the force factor.

\begin{figure}[!ht]
	\begin{center}
	\begin{circuitikz}%[scale=0.8, transform shape]
		\draw
  	 	 (2,2) node[gyrator] (T) {}
    (T.A2)  -- + (-2,0) coordinate (aux1)
            to[V^=$e_{g}$]    (aux1  |- T.A1) 
            to[generic,l=$R_{e}$] (T.A1)           
    (T.B1)  to[R=$R_{ms}$] + (2,0)
            to[C=$C_{ms}$] + (2,0)
            to[L=$M_{ms}$] + (2,0)
            to[C=$C_{box}$] + (0,-2) |- (T.B2)
    (T.base) node{Bl:1};
	\end{circuitikz}
	\caption{Equivalent circuit of a driver in the box}
	\label{fig:eq_circ}
	\end{center}
\end{figure}

From equation \ref{eq:TRF_disp} and the equivalent circuit presented by figure \ref{fig:eq_circ} the displacement transfer function of the system in the box can be expressed by equation \ref{eq:TRF_dispbox}. $C_{msb}$ is the equivalent compliance of the system in the box such as $C_{msb} = (C_{ms} * C_{box}) / (C_{ms} + C_{box})$ and $e_{gb}$ the input voltage.

\begin{equation}
x_{b}(\omega ) = e_{gb}\cdot \frac{C_{msb}Bl}{R_{e}}\cdot \frac{1}{(j\omega /\omega _{s})^{2}+(1/Q_{ts})(j\omega /\omega _{s})+1}
\label{eq:TRF_dispbox}
\end{equation}
\myequations{Displacement transfer function in the box}

To correct the effect of the box, an input voltage for which $x(\omega )$ = $x_{b}(\omega ) $ can be calculated as given in equation \ref{eq:V_corr} where $K_{ms}$ and $K_{msb}$ are the stiffness in free air and in the box respectively, such as $K_{ms}$ = $\frac{1}{C_{ms}}$.

\begin{equation}
e_{gb} = e_{g}\cdot \frac{C_{ms}}{C_{msb}} = e_{g}\cdot \frac{K_{msb}}{K_{ms}}
\label{eq:V_corr}
\end{equation}
\myequations{Voltage correction for the box}

This voltage can be expressed as a function of the resonance frequency given by equation \ref{eq:fs}, where $M_{ms}$ is the moving mass of the driver. 

\begin{equation}
f_{s} = \frac{1}{2 \pi \sqrt{\frac{M_{ms}}{K_{ms}}}}
\label{eq:fs}
\end{equation}
\myequations{Resonance frequency of a speaker}

Expressing the voltage as a function of the resonance frequency makes the compensation easier to estimate, as the calculation of the Thiele \& Small parameters is no longer required. From a simple electrical measurement of the impedance, which is the ratio of the output voltage by the output current $Z = U_{s}/I_{s}$, the resonance frequency can be found at the maximum of the impedance. 

\begin{equation}
e_{gb} = e_{g}\cdot \frac{(2 \pi f_{s})^{2}}{(2 \pi f_{b})^{2}}
\label{eq:V_corr_fs}
\end{equation}
\myequations{Voltage correction for the box as a function of resonance frequency}

Equation \ref{eq:V_corr_fs} gives the compensated voltage for the box, where $ f_{s}$ is the resonance frequency of the driver in free air and $ f_{b}$ the resonance frequency of the driver in the box. This implies that two measurements are required: one of the driver in free air, and one of the driver in the box. \\
Now that the theory is expressed, this assumption needs to be verified. To do so, the impedance of the driver and the displacement transfer function of the driver's membrane have been measured on the 10 cm woofer by using the Klippel system, and a laser pointing on the driver's membrane estimates the displacement. \\

Figure \ref{fig:imp_comp} presents the comparison of the measured impedances for different conditions: the initial measures in free air and in the box to estimate the voltage compensation to apply, and the results of the new measurements when the voltage compensations is calculated from the $K_{ms}$ (measured using Klippel's laser method for Thiele \& Small parameters estimation) and the $f_{s}$ (measured at the maximum of the impedance curves) have been applied. \\
When looking at these results, few differences between the impedances can be found. However, when looking at the displacement transfer functions presented in figure \ref{fig:disp_comp}, interesting results are observed. \\

\begin{minipage}{0.5\textwidth}
\begin{center}
	\includegraphics[width=0.9\textwidth]{RoomComp/Imp_Comps} 
    \captionsetup{hypcap=false} 
	\captionof{figure}{Comparison of measured impedance} 
	\label{fig:imp_comp}
\end{center}
\end{minipage}
\begin{minipage}{0.5\textwidth}
\begin{center}
	\includegraphics[width=0.9\textwidth]{RoomComp/Disp_Comps} 
    \captionsetup{hypcap=false} 
	\captionof{figure}{Comparison of measured displacement TRF} 
	\label{fig:disp_comp}
\end{center}
\end{minipage}
\vspace{0.1cm}

Comparing the free air and in box measurement shows that up to 50 Hz the membrane's displacement is greater in the box, generating the "bass boost" effect observed on figure \ref{fig:Load_Compa}. The displacement is then larger in free air than in the box, except at the driver's resonance.\\
Using the voltage correction calculated either by the ratio of resonance frequencies $f_{s}$ and $f_{b}$ or the ratio of stiffness $K_{ms}$ lowers the membrane's displacement in the box, although the attenuation provided by the ratio of resonance frequencies is overestimated: the membrane's displacement is too attenuated in the 20 - 500 Hz frequency band. The attenuation provided by the ratio of stiffness gives promising results, particularly in the "bass boost" frequency band (20 - 50 Hz) and around the resonance, where the displacement transfer function in the box gets close to the one measured in free air. \\

The mismatches between the free air and voltage compensated curves are most likely due to the approximations made on the box's behavior. The effect of the box of the driver is much more complicated than a change on the mechanical compliance, especially since leakage isn't taken in account.  \\

Although using voltage compensation to ease-off the influence of the box isn't the most accurate solution, it can be considered as a fast solution for performing measurement of loudspeakers in the linear domain using the tetrahedral box; without using a Compensation curve. Furthermore, it is possible to extrapolate these measurements to the far field of the driver by using the dB Keele method described in section \ref{sec:dBKeele}.

\subsection{Measures in the tetrahedral box}

\begin{minipage}{0.6\textwidth}
Finally, measurements of the 10 cm woofer (see appendix \ref{spkrlib:10cm}) have been made in the tetrahedral box and compensated using a measurement done using the Near Field Scanner, and a measure done on-axis at 5 cm of the speaker.\\
The distance microphone - speaker in the box is $D_{ms}$ = 5 cm. Results are presented in figure \ref{fig:tet_res}.\\

The "bass boost" is very visible on the uncompensated measurement, and have a huge impact on the results in low frequency. When using a near field measurement to compensate for the box effects, the results are acceptable. Because of the near field interference effects some variations can be observed, particularly in the 130 - 750 Hz range. \\
When using a NFS result to perform the Room Compensation, the results are very good: the NFS allows to ease off the near field effects, and provides accurate results over the complete measurement range. 
\end{minipage}
\begin{minipage}{0.4\textwidth}
\begin{center}
	\includegraphics[width=\textwidth]{RoomComp/tetbox_res} 
    \captionsetup{hypcap=false} 
	\captionof{figure}{Measurement in the tetrahedral box} 
	\label{fig:tet_res}
\end{center}
\end{minipage}

\section{Other works}

In this work's context, application note describing how to use the Complex Room Compensation module have been redacted and is \href{https://www.dropbox.com/s/vqfezho53hyu7xg/AN_00_Complex_Room_Compensation.pdf?dl=0}{available for download \underline{here}}.\\
 Another task was to implement the possibility of selecting an user-defined reference curve to perform the Room Compensation; allowing Klippel customers to use other references than a NFS result to ease-off a room's effects on an acoustic measurement.

\section{Discussion}

In this chapter, several methods for compensating the effect of the boundaries on an acoustic measurement have been  investigated. Table \ref{tab:room_comp} compares all of these solutions, and evidences the advantages and drawbacks of each method. 

\vspace{0.5cm}

\begin{table}[ht]
\centering
\begin{tabular}{ p{2.3cm} *{4}{p{2cm}} *{2}{p{2.3cm}}}
\toprule
                                                              & \thead{Fast \\ measure} & \thead{Shielding}   & \thead{low \\ frequency  \\accuracy} & \thead{high \\ frequency \\accuracy}  & \thead{off-axis \\ measure} & \thead{multiple \\ diaphragm \\ systems  \\measure} \\\midrule
\makecell{nearfield \\ reference}          &		\makecell{\makecell{\textbf{X}}	}			&                 	      &       	\makecell{\textbf{X}}		          &        		      &      	\makecell{\textbf{X}}		           &                                                \\\midrule
\makecell{NFS \\ reference}                    &			\makecell{\textbf{X}}						&                       &     	\makecell{\textbf{X}}		        &  \makecell{\textbf{X}}		 &       	\makecell{\textbf{X}}		        &       	\makecell{\textbf{X}}		        \\\midrule
\makecell{D.B.Keele \\ extrapolation}     &				\makecell{\textbf{X}}					&                       & 	\makecell{\textbf{X}}		       &                                 &                                                   &         	\makecell{\textbf{X}}		           \\\midrule
\makecell{box - voltage \\ compensation}    &			\makecell{\textbf{X}}						&   \makecell{\textbf{X}}		    &                                  & \makecell{\textbf{X}}		        &                 &                                                \\ \midrule
\makecell{box - nearfield \\ compensation}       &			\makecell{\textbf{X}}						&   \makecell{\textbf{X}}		      &   						      &     \makecell{\textbf{X}}		 &                   &                                                \\\midrule
\makecell{box - NFS \\ compensation}                &				\makecell{\textbf{X}}					&    \makecell{\textbf{X}}		      &      	\makecell{\textbf{X}}		       &  \makecell{\textbf{X}}		&               &                                                \\\bottomrule
\end{tabular}
\caption{Comparison of different Room Compensation methods}
\label{tab:room_comp}
\end{table}

\vspace{0.5cm}

Depending on the user's intentions and requirements, specific solutions should be preferred to other. For example to attest the quality of a speaker at the end of a production line, solutions providing shielding against changes on the measurement environment and accuracy over the whole frequency spectrum should be preferred; meanwhile an utilization for research and development to assess a speaker array should perform the measurement in free air and use the compensation by a NFS result: the shielding isn't required, but a good accuracy over the whole frequency range and the possibility to measure multiple diaphragms systems are relevant to this application. 


%--------------------------------------------------------------
%	CONCLUSION
%--------------------------------------------------------------
\chapter*{Conclusion}
\addcontentsline{toc}{chapter}{Conclusion}

Several investigations have been conducted in the context of this report and gave significant results. \\

For the NFS measurements in baffle, several measurements have been conducted, allowing to diagnose the problems with a prototype designed to perform such measurements. It has been evidenced that the wooden plate will influence the measurement; and that the baffle is subject to rattling if the clamping system isn't correctly tightened.  \\
The development of an elliptical measurement grid have shown that for measurement not using the field separation method, measuring closer to the driver gives more accurate results by minimizing the impact of the room modes.\\
This work could be continued by generating other shape of grids; and by implementing the "vibration diagnosis" method described in the discussion of the \textit{Estimation of baffle influence} section. \\

The optimization of the NFS measurement time gave interesting insights on how to exploit the symmetry conditions on the complex coefficients of the spherical wave expansion, and lead to the implementation of a symmetry check algorithm. This work allows the users of the Near Field Scanner to perform faster measurements, or to speed up the processing time of the results. \\
The continuation of this study is the implementation of the symmetry check algorithm in the NFS software, and the generation of relevant grids for each symmetry type. \\

Finally, the work on the Room Compensation methods gave an overview of different methods to obtain Reference Curves and ease the effects of the boundaries off an acoustic measurement. A comparative table have been created, summarizing all the methods studied during this work, and allowing to compare them and pinpoint the most relevant solution for a given problematic. \\

On a more personal note, working at Klippel GmbH have been a very enjoyable experience and gave me more insight about the real functioning of a company and what to expect from my future. I am thrilled to have had the opportunity to conduct this work, and very satisfied with the variety of the tasks I was assigned to, which included programming, measuring, and research to draw conclusion from my experiments. \\

Thank you for reading my work.

%--------------------------------------------------------------
%	BIBLIOGRAPHY
%--------------------------------------------------------------
\renewcommand{\bibname}{References}
\addcontentsline{toc}{chapter}{References}
\bibliography{bibli}
\bibliographystyle{abbrv}

\nocite{*}

%--------------------------------------------------------------
%	APPENDIX
%--------------------------------------------------------------

\begin{appendices}

		% Additional curves
\chapter{Additional figures}
\pagenumbering{roman}
This appendix aims to present additional figures that could not be presented in the corpus of this report because of their size. To simplify the navigation, the titles of the section are the same as the titles of the chapters this figures are complemented with. 


\section{Baffle measurements}
\subsection{Estimation of baffle influence}
\label{Curves:BaffleInfluence}

\begin{center}
	\includegraphics[width=0.9\textwidth]{Appendix/Vib_RadPow}
    \captionsetup{hypcap=false}
    \captionof{figure}{Radiated Powers at 1m for different measurement points on the baffle}
    \label{Curves:Baffle_RadPow}
\end{center}


\section{Measurement time optimization}

\subsection{10 cm woofer}
\label{Curves:10cm}

\begin{minipage}{0.5\textwidth}
\begin{center}
	\includegraphics[width=.9\textwidth]{Sym/contour} 
    \captionsetup{hypcap=false} 
	\captionof{figure}{Contour plot of the woofer at 10 m, centered on-axis, horizontal plane.} 
	\label{fig:contour_10cm}
\end{center}
\end{minipage}
\begin{minipage}{0.5\textwidth}
\begin{center}
	\includegraphics[width=0.9\textwidth]{Sym/10cm_RadPow_Rot}
    \captionsetup{hypcap=false}
    \captionof{figure}{10 cm woofer, Radiated Powers: rotational symmetry}
\end{center}
\end{minipage}

\begin{minipage}{0.5\textwidth}
\begin{center}
	\includegraphics[width=0.9\textwidth]{Sym/10cm_RadPow_B}
    \captionsetup{hypcap=false}
    \captionof{figure}{10 cm woofer, Radiated Powers: baffle symmetry}
\end{center}
\end{minipage}
\begin{minipage}{0.5\textwidth}
\begin{center}
	\includegraphics[width=0.9\textwidth]{Sym/10cm_RadPow_Dps}
    \captionsetup{hypcap=false}
    \captionof{figure}{10 cm woofer, Radiated Powers: dual plane symmetry}
\end{center}
\end{minipage}

\begin{minipage}{0.5\textwidth}
\begin{center}
	\includegraphics[width=0.9\textwidth]{Sym/10cm_RadPow_SpsLR}
    \captionsetup{hypcap=false}
    \captionof{figure}{10 cm woofer, Radiated Powers: single plane (left/right) symmetry}
\end{center}
\end{minipage}
\begin{minipage}{0.5\textwidth}
\begin{center}
	\includegraphics[width=0.9\textwidth]{Sym/10cm_RadPow_SpsTB}
    \captionsetup{hypcap=false}
    \captionof{figure}{10 cm woofer, Radiated Powers: single plane (top/bottom) symmetry}
\end{center}
\end{minipage}

\subsection{Oval speaker}
\label{Curves:oval}

\begin{minipage}{0.5\textwidth}
\begin{center}
	\includegraphics[width=\textwidth]{Sym/contour_oval}
    \captionsetup{hypcap=false}
    \captionof{figure}{Oval speaker, Radiated Powers: zoom on 6 - 20 kHz}
    \label{Curves:oval_contour}
\end{center}
\end{minipage}
\begin{minipage}{0.5\textwidth}
\begin{center}
	\includegraphics[width=0.9\textwidth]{Sym/Oval_RadPow_Rot}
    \captionsetup{hypcap=false}
    \captionof{figure}{Oval speaker, Radiated Powers: rotational symmetry}
\end{center}
\end{minipage}

\begin{minipage}{0.5\textwidth}
\begin{center}
	\includegraphics[width=0.9\textwidth]{Sym/Oval_RadPow_B}
    \captionsetup{hypcap=false}
    \captionof{figure}{Oval speaker, Radiated Powers: baffle symmetry}
\end{center}
\end{minipage}
\begin{minipage}{0.5\textwidth}
\begin{center}
	\includegraphics[width=0.9\textwidth]{Sym/Oval_RadPow_Dps}
    \captionsetup{hypcap=false}
    \captionof{figure}{Oval speaker, Radiated Powers: dual plane symmetry}
\end{center}
\end{minipage}

\begin{minipage}{0.5\textwidth}
\begin{center}
	\includegraphics[width=0.9\textwidth]{Sym/Oval_RadPow_SpsLR}
    \captionsetup{hypcap=false}
    \captionof{figure}{Oval speaker, Radiated Powers: single plane (left/right) symmetry}
\end{center}
\end{minipage}
\begin{minipage}{0.5\textwidth}
\begin{center}
	\includegraphics[width=0.9\textwidth]{Sym/Oval_RadPow_SpsTB}
    \captionsetup{hypcap=false}
    \captionof{figure}{Oval speaker, Radiated Powers: single plane (top/bottom) symmetry}
\end{center}
\end{minipage}

\subsection{2-way "diagonal" speaker}
\label{Curves:2way}

\begin{minipage}{0.5\textwidth}
\begin{center}
	\includegraphics[width=\textwidth]{Sym/Contour_BnO}
    \captionsetup{hypcap=false}
    \captionof{figure}{Contour plot of the diagonal speaker at 10 m, centered on-axis, horizontal plane.}
    \label{Curves:2way_contour}
\end{center}
\end{minipage}
\begin{minipage}{0.5\textwidth}
\begin{center}
	\includegraphics[width=0.9\textwidth]{Sym/BnO_RadPow_Rot}
    \captionsetup{hypcap=false}
    \captionof{figure}{Diagonal speaker, Radiated Powers: rotational symmetry}
\end{center}
\end{minipage}

\begin{minipage}{0.5\textwidth}
\begin{center}
	\includegraphics[width=0.9\textwidth]{Sym/BnO_RadPow_B}
    \captionsetup{hypcap=false}
    \captionof{figure}{Diagonal speaker, Radiated Powers: baffle symmetry}
\end{center}
\end{minipage}
\begin{minipage}{0.5\textwidth}
\begin{center}
	\includegraphics[width=0.9\textwidth]{Sym/BnO_RadPow_Dps}
    \captionsetup{hypcap=false}
    \captionof{figure}{Diagonal speaker, Radiated Powers: dual plane symmetry}
\end{center}
\end{minipage}

\begin{minipage}{0.5\textwidth}
\begin{center}
	\includegraphics[width=0.9\textwidth]{Sym/BnO_RadPow_SpsLR}
    \captionsetup{hypcap=false}
    \captionof{figure}{Diagonal speaker, Radiated Powers: single plane (left/right) symmetry}
\end{center}
\end{minipage}
\begin{minipage}{0.5\textwidth}
\begin{center}
	\includegraphics[width=0.9\textwidth]{Sym/BnO_RadPow_SpsTB}
    \captionsetup{hypcap=false}
    \captionof{figure}{Diagonal speaker, Radiated Powers: single plane (top/bottom) symmetry}
\end{center}
\end{minipage}

\section{Room Compensation}

\subsection{Measurements in the tetrahedral box}

\subsubsection{Influence of microphone position}
\label{Curves:InfluMicPos}


\begin{center}	
	\includegraphics[scale=0.76,angle=90]{RoomComp/MicPos_TRF} 
	\captionsetup{hypcap=false} 
	\captionof{figure}{Transfer functions for all microphones positions} 
	\label{fig:micpos_TRF_big}
\end{center}


		% Speaker library
\chapter{Speaker library}
\label{chap:spk_lib}

\section{10 cm woofer}
\label{spkrlib:10cm}

\begin{minipage}{0.6\textwidth}
This speaker is a 10 cm (4'') woofer from PSS, designed for automotive applications. A picture is shown in figure \ref{fig:spkr_lib_10cm}\\ 
The Thiele \& Small  parameters of this speaker have been measured using the Klippel scanning system and are presented in table \ref{tab:TnS_10cmWoofer}. 
\end{minipage}
\begin{minipage}{0.4\textwidth}
\begin{center}
	\includegraphics[scale=1]{Appendix/Round_Spkr}
    \captionsetup{hypcap=false}
    \captionof{figure}{10 cm woofer}
    \label{fig:spkr_lib_10cm}
\end{center}
\end{minipage}




\begin{table}[ht]
\centering
\begin{tabular}{|c|c|c|c|c|c|c|c|c|c|}
\hline
\begin{tabular}[c]{@{}c@{}}Re \\{[}$\Omega${]}\end{tabular} & \begin{tabular}[c]{@{}c@{}}Fs\\{[}Hz{]}\end{tabular} & \begin{tabular}[c]{@{}c@{}}Mms\\  {[}g{]}\end{tabular} & \begin{tabular}[c]{@{}c@{}}Cms\\{[}mm/N{]}\end{tabular} & \begin{tabular}[c]{@{}c@{}}Rms\\  {[}kg/s{]}\end{tabular} & Bl    & Qms   & Qes   & Qts   & \begin{tabular}[c]{@{}c@{}}Vas\\  {[}L{]}\end{tabular} \\ \hline
3.28                                                                        & 145.8                                                                 & 5.492                                                                 & 0.217                                                                    & 1.619                                                                    & 3.197 & 3.108 & 1.614 & 1.063 & 1.89                                                                  \\ \hline
\end{tabular}
\caption{T \& S parameters of 10cm woofer}
\label{tab:TnS_10cmWoofer}
\end{table}

\section{Oval speaker}
\label{spkrlib:oval}

\begin{minipage}{0.6\textwidth}
This speaker is a 23 cm x 15 cm fullband speaker. A picture is shown in figure \ref{fig:spkr_lib_Oval}.\\
The T\&S parameters of the speaker were not given and haven't been measured.

\end{minipage}
\begin{minipage}{0.4\textwidth}
\begin{center}
	\includegraphics[scale=1]{Appendix/Oval_Spkr}
    \captionsetup{hypcap=false}
    \captionof{figure}{Oval speaker}
    \label{fig:spkr_lib_Oval}
\end{center}
\end{minipage}


\section{2-way diagonal speaker}
\label{spkrlib:BnO}

\begin{minipage}{0.6\textwidth}
This 2-way speaker presents the particularity of having its tweeter and woofer mounted diagonally as shown by figure \ref{fig:2wayDiagSpkr}. The measurements have been focused on the tweeter of this speaker, and the reference axis for the measure have been setted along the "measurement axis" showed on the side view in figure \ref{fig:2wayDiagSpkr}.

\end{minipage}
\begin{minipage}{0.4\textwidth}
\begin{center}
	\includegraphics[scale=0.2]{Sym/Spkr_BmO}
    \captionsetup{hypcap=false}
    \captionof{figure}{2-way diagonal speaker}
    \label{fig:2wayDiagSpkr}
\end{center}
\end{minipage}



	% More baffle prototype images
\chapter{Baffle Prototype pictures}
\label{Chap:baffle_pics}
This chapter presents pictures of the baffle prototype, and particularly of the clamping system. \\

\begin{minipage}{0.5\textwidth}
\begin{center}
	\includegraphics[width=0.9\textwidth]{Appendix/baffle_back}
    \captionsetup{hypcap=false}
    \captionof{figure}{Back of the baffle}
    \label{fig:baffleback}
\end{center}
\begin{center}
	\includegraphics[width=0.9\textwidth]{Appendix/baffle_clamps}
    \captionsetup{hypcap=false}
    \captionof{figure}{Closeup of the clamping system}
    \label{fig:baffleclamp}
\end{center}
\end{minipage}
\begin{minipage}{0.5\textwidth}
\begin{center}
	\includegraphics[width=0.9\textwidth]{Appendix/baffle_side}
    \captionsetup{hypcap=false}
    \captionof{figure}{Side of the baffle}
    \label{fig:baffleside}
\end{center}
\end{minipage}

		


\chapter{SciLab implementation of symmetry verification} 
\label{Chap:imple_sym}
The following code shows how these equations have been implemented in SciLab.\\
\textit{\textbf{NOTE}: This program have been simplified by removing some functions, the aim being to demonstrate how the algorithm calculating the coefficients based on symmetry conditions works.}

\lstset{language=SCilab} 
\begin{lstlisting}
//------------------------------------------------MAIN FUNCTION
function OnRun()
// Loading the datas
Cout = load('Cout.txt');
Cin  = load('Cin.txt');
f    = load('f.txt');

N = sqrt(size(Cout,1))-1; // Number of points
// Constants 
P0  = 10^(-12);		      //reference sound power
rho = 1.204;		         //density of the air
c   = 343;		            //speed of sound
k   = 2*%pi*f'/c;	         //wavenumber k
A   = 4*%pi*radius^2; 
// Rotational symmetry (example)
[Csym_ROT, Crest_DPS, powOut_rot, powRestOut_rot] = CheckSymAll(Cout, N, f, 'ROT');
ROT_TotOut_FF  = [f' 10*log10(powOut_rot' ./(rho*c*k^2)/P0)];        // Total outgoing power 
ROT_OutRest_FF = [f' 10*log10(powRestOut_rot' ./(rho*c*k^2)/P0)];    // Rest of outgoing power
ROT_SymFactor_FF = [f' (abs(powOut_rot./(powOut_rot+powRestOut_rot)))'];    // Symmetry factor
endfunction // OnRun

//--------------------------------------------Find the coeffs for different symmetries
function [Csym, Crest, power_sym, power_rest] = CheckSymAll(C, N, f, SymType)
   // CheckSymAll: calculates the coeff according the different symmetry conditions
   // Inputs:     C: coefficients to find symmetry on 
   //             N: number of iterations 
   //             f: frequency vector
   //             SymType: a string with the symmetry types. 
   //	          Allowed values are: 'ROT', 'DPS', 'BS', 'SPS_LR', 'SPS_TB' 
   //                      representing respectively Rotational Symmetry, Dual Plane Symmetry, 
   //                      Baffle Symmetry, Single Plane Symmetry (Left/Right),
   //					   Single Plane Symmetry (Top/Bottom).
   // Outputs:    Csym: coefficients of the symmetry condition
   //             Crest: rest of coefficients = all coeffs - Csym
   //             power_sym: power of the symmetry coefficients
   //             power_rest: power of the rest coeffcients   
   
   // Initialization of matrices (all coeffs, left and right coefficients)
   Csym = zeros(size(C,'r'),size(C,'c'));
   Csym_l = zeros(size(C,'r'),size(C,'c'));  
   Csym_r = zeros(size(C,'r'),size(C,'c'));
   
   // Calculation of coefficient according to symmetry condition assumed
   select SymType
   case 'ROT' 
      [nm] = calc_nm_index(N*ones(f), 1);                      
      [n, m] = order_suborder(nm(:,1));   // Finding the equivalent n and m values   
      
      // Finding the equivalent index value for m=0
      index = GetIndex(n, m);      
      Csym(index,:) = C(index,:);   // If m=0 we put the value in C in the Csym
      // Calculation of rest 
      Crest = C - Csym;
   case 'DPS'
      [nm] = calc_nm_index(N*ones(f), 5);                
      [n, m] = order_suborder(nm(:,1));   // Finding the equivalent n and m values
      index_row = GetIndex(n, m);

      Csym_l(index_row,:) = C(index_row,:);  // Filling the Coeffs of the left with the correct data
      
      // Finding the equivalent m value for the right side
      for i = 1:size(m,'r')
         m_r(i) = -1*m(i);
         index_r(i) = GetIndex(n(i), m_r(i));
         index_l(i) = GetIndex(n(i), m(i));
         if isequal(index_l(i),index_r(i))==%F then
            Csym_r(index_r(i),:) = (1)*C(index_l(i),:);
         end
      end
      Csym = Csym_l + Csym_r;
      // Calculation of rest 
      Crest = C - Csym;
   case 'BS'
      [nm] = calc_nm_index(N*ones(f), 2);    // For baffle
      [n, m] = order_suborder(nm(:,1));   // Finding the equivalent n and m values      
      index_row = GetIndex(n, m);                // Index of relevant rows for the Coeffs         
      Csym(index_row,:) = C(index_row,:);  
      // Calculation of rest 
      Crest = C - Csym;
   case 'SPS_LR'
      [nm] = calc_nm_index(N*ones(f), 4);         // Indexes of left side
      [n, m] = order_suborder(nm(:,1));   // Finding the equivalent n and m values
      index_left = GetIndex(n, m);
            
      m_right = -1.*m;                          // Indexes of right side
      weights = ((-1)^(m));
      index_right = GetIndex(n, m_right);
      
      [nm] = calc_nm_index(N*ones(f), 1);         // Indexes of middle (m=0)
      [n, m] = order_suborder(nm(:,1));   // Finding the equivalent n and m values
      index_middle = GetIndex(n, m);
      
      Csym_l(index_left,:) = C(index_left,:);      
      for i=1:size(C,'c')
         Csym_r(index_right,i) = Csym_l(index_left,i).*weights;   // Weighting the coefficients
      end
      Csym_m(index_middle,:) = C(index_middle,:);      
      Csym_r2(index_right,:) = C(index_right,:);
      for i=1:size(C,'c')
         Csym_l2(index_left,i) = Csym_r2(index_right,i)./weights;
      end 
      
      avg_Left = (Csym_l + Csym_l2)/2;
      avg_Right = (Csym_r + Csym_r2)/2;
      Csym = Csym_m + avg_Left + avg_Right;
      Crest = C - Csym;
   case 'SPS_TB'
      [nm] = calc_nm_index(N*ones(f), 4);         // Indexes of left side
      [n, m] = order_suborder(nm(:,1));   // Finding the equivalent n and m values
      index_left = GetIndex(n, m);
      
      m_right = -1*m;
      index_right = GetIndex(n, m_right);
      
      [nm] = calc_nm_index(N*ones(f), 1);         // Indexes of middle (m=0)
      [n, m] = order_suborder(nm(:,1));   // Finding the equivalent n and m values
      index_middle = GetIndex(n, m);
      
      Csym_l(index_left,:) = C(index_left,:);
      Csym_r(index_right,:) = Csym_l(index_left,:);   // No weighting is needed for the T/B symmetry
      Csym_m(index_middle,:) = C(index_middle,:);      
      Csym_r2(index_right,:) = C(index_right,:);
      Csym_l2(index_left,:) = Csym_r2(index_right,:);
      
      avg_Left = (Csym_l + Csym_l2)/2;
      avg_Right = (Csym_r + Csym_r2)/2;      
      Csym = Csym_m + avg_Left + avg_Right;
      Crest = C - Csym;
   end
   
   // Calculation of power
   power_sym  = sum(abs(Csym).^2,1);    // Estimating the power of sym
   power_rest = sum(abs(Crest).^2,1);   // Estimating the power of rest
endfunction
\end{lstlisting}


		% GANTT CHART
\chapter{Organization of the work}


The following presents the work schedule I have followed, what work have been done and when it have been started and finished. This is presented in the following GANTT chart.

\begin{center}
	\includegraphics[width=\textwidth]{Appendix/Plan1} 
\end{center}


		% Application Notes
\chapter{Measurement \& results databases}

If the readers of this report wish to see the measurement results presented in this work, I have uploaded the databases of each section on my dropbox. \\
The baffle analysis database shows the measurements at different points on the baffle, the comparison between the Speaker only and the Speaker+Wood measurements, and the comparison of the spherical and elliptical grid. \href{https://www.dropbox.com/s/we7o95gzfl58f5d/Baffle_Measurements.kdbx?dl=0}{Please click \underline{here} to download.} \\

The symmetry archive contains the measurements of the 10 cm woofer, the oval speaker, and the diagonal speaker; and the results of the symmetry check.  \href{https://www.dropbox.com/s/zyhlq3nanyxap8a/Symmetry_Check.zip?dl=0}{Please click \underline{here} to download.}  \\

The Room Compensation database gives the results of the different Room Compensation methods, and the measurements of the box's optimal microphone position. \href{https://www.dropbox.com/s/0xq1em9ftyg0ncr/Room_Compensation.kdbx?dl=0}{Please click \underline{here} to download.} \\

To visualize the results, the reader should download and install the Klippel software dB-Lab viewer, available \href{https://www.klippel.de/dm/}{\underline{here}}, under the path \textit{software} /  \textit{dB-Lab Viewer} / \textit{RnD 210 Viewer}.


\end{appendices}

\end{document}

% USELESS SHIT (but I spent time on it so I keep it mwahaha)

%\begin{equation}
%C_{mn_{Left}} = \begin{pmatrix}
%    C_{0,0}(f_{1})    & ... & ... & C_{0,0}(f_{n})   \cr 
%    0   & ... & ... & 0 \cr
%    C_{0,1}(f_{1})    & ... & ... & C_{0,1}(f_{n})   \cr
%    0       & ... & ... & 0   \cr
%    C_{-2,2}(f_{1})    & ... & ... & C_{-2,2}(f_{n})  \tikzmark{aarrowtop} \cr
%    ...               & ... & ... & ...               \cr \end{pmatrix} \qquad
%C_{mn_{Right}} =  \begin{pmatrix}
%    0    & ... & ... & 0   \cr 
%    0    & ... & ... & 0 \cr
%    0    & ... & ... & 0   \cr
%    0    & ... & ... & 0   \cr
%    0    & ... & ... & 0   \cr
%    0    & ... & ... & 0  \cr 
%    0    & ... & ... & 0   \cr 
%    0    & ... & ... & 0  \cr 
%  \tikzmark{aarrowbottom}  C_{2,2}(f_{1})    & ... & ... & C_{2,2}(f_{n})\cr 
%    ...               & ... & ... & ...               \cr \end{pmatrix}
%\label{eq:matrix_DPS}
%\end{equation}
%\tikz[overlay,remember picture] {
%  \draw[->] ([xshift=1.5ex]aarrowtop) -- ([xshift=-1.5ex,yshift=0.5ex]aarrowbottom)
%  node[midway,above] {\scriptsize $=$};
%}


%
%\begin{minipage}{0.5\textwidth}
%\begin{center}
%	\includegraphics[width=0.8\textwidth]{Appendix/Spherical_Harmo}
%    \captionsetup{hypcap=false}
%    \captionof{figure}{Spherical Harmonics}
%    \label{fig:Sph_Harmo}
%\end{center}
%\end{minipage}
%\begin{minipage}{0.5\textwidth}
%The solution of the three-dimensional wave equation in spherical coordinates can be described by a spherical wave expansion given by equation \ref{eq:sphwvexp}.The free-field transfer function of a loudspeaker can be expressed by only considering the direct sound (outgoing wave) in equation \ref{eq:sphwvexp} and is expressed in equation \ref{eq:ffHf}.\\
%
%Figure \ref{fig:Sph_Harmo} presents a visual representation of the first few spherical harmonics, and their orders (n,m) (\textit{all figures presented in this subsection are courtesy of W. Klippel, from the presentation \cite{lect2018}}). 
%\end{minipage}
%\vspace{0.4cm}
%
%If the DUT's geometry presents certain symmetries, its radiated sound field will exhibit the same symmetries. This property allows to decrease the number of coefficients, thus reducing the scanning effort. The different symmetries under study are listed below.\\
%
%\begin{minipage}{0.7\textwidth}
%\textbf{Rotational symmetry} (see figure \ref{fig:rotsym}) is observed for many circular drivers and is the simplest type of symmetry as a lot of coefficients are vanishing according to the relation given by equation \ref{eq:Crotsym} \cite{aeshs}.
%
%\begin{equation}
%C_{mn} = 0 \; \; \; \; \forall \; m \neq 0
%\label{eq:Crotsym}
%\end{equation}
%\myequations{Coefficients for rotational symmetry}
%
%From the measurements, the coefficients are obtained in matrix form where the columns represent the $nm$ orders and the rows the frequency, shown in equation \ref{eq:matrix_cmn}.
%
%%-------------------------- Coeffs matrix
%\begin{equation}
%\label{eq:matrix_cmn}
%\resizebox{.9\textwidth}{!}{$\displaystyle
%  C_{mn} = \qquad \bordermatrix{   &   & \tikzmark{harrowleft} &  &  &  &  &  \tikzmark{harrowright}  &  &&  \cr
%                    \tikzmark{varrowtop} & C_{0,0} &  C_{-1,1} & C_{0,1} & C_{1,1} & C_{-2,2} & C_{-1,2} & C_{0,2} & C_{1,2} & C_{2,2} & ...  \cr                   								 
%                    \tikzmark{varrowbottom} & ... & ... & ... & ... & ... & ... & ... & ... & ... & ... \cr
%                    }
%$}
%\end{equation}
%\myequations{Matrix of coefficients}
%
%\tikz[overlay,remember picture] {
%  \draw[->] ([yshift=1.1ex,xshift=1ex]harrowleft) -- ([yshift=1.1ex]harrowright)
%            node[midway,above] {\scriptsize $m,n$};
%  \draw[->] ([yshift=1ex,xshift=-1ex]varrowtop) -- ([xshift=-1ex]varrowbottom)
%            node[near end,left] {\scriptsize $f$};
%}
%%-------------------------- Coeffs matrix
%\end{minipage}
%\begin{minipage}{0.3\textwidth}
%\begin{center}
%	\includegraphics[width=\textwidth]{Appendix/Rot_Sym}
%    \captionsetup{hypcap=false}
%    \captionof{figure}{Rotational symmetry}
%    \label{fig:rotsym}
%\end{center}
%\end{minipage}
%
%All m $\neq$ 0 coefficients are vanishing, meaning that only m = 0, n = n+1 coefficients have to be estimated. \\
%
%Once the coefficients have been estimated, the symmetry factor S can be deduced from the ratio of coefficients satisfying the symmetry condition by the total coefficients. For rotational symmetry, $S_{rs}$ is given by equation \ref{eq:Srs} \cite{aeshs}.
%
%\begin{equation}
%S_{rs} = 1-\frac{\sum_{n=1}^{N}\sum_{s=1}^{n}\left | C_{sn} \right |^{2}}{\sum_{n=0}^{N}\sum_{m=-n}^{n} C_{mn}^{2}}
%\label{eq:Srs}
%\end{equation}
%\myequations{Symmetry factor for rotational symmetry}
%
%
%\begin{minipage}{0.7\textwidth}
%\textbf{Dual plane symmetry} (see figure \ref{fig:dualpsym}) is mostly observed in 2-way closed box systems, and make half of the coefficients vanish according to the relation given by equation \ref{eq:Cdpsym} \cite{aeshs}. 
%
%\begin{equation}
%C_{mn}(f) = C_{-mn}(f)R_{m}(f) \; \; \; \;  m = 2s; \; s =1, 2, 3...
%\label{eq:Cdpsym}
%\end{equation}
%\myequations{Coefficients for dual plane symmetry}
%Where Rm represents a complex symmetry parameter, expressed in equation \ref{eq:Rm} \cite{aeshs}.
%
%\begin{equation}
%R_{m}(f) = (-1)^{m+1}(\textup{sin}(m \phi _{s}(f)) + i\textup{cos}(m \phi _{s}(f)))^{2}
%\label{eq:Rm}
%\end{equation}
%\myequations{Complex symmetry parameter}
%Where $\phi _{s}$ is the symmetry axis; and $m$ the suborder of the coefficient under study.
%\end{minipage}
%\begin{minipage}{0.3\textwidth}
%\begin{center}
%	\includegraphics[width=0.8\textwidth]{Appendix/Dual_Plane_Sym}
%    \captionsetup{hypcap=false}
%    \captionof{figure}{Dual Plane symmetry}
%    \label{fig:dualpsym}
%\end{center}
%\end{minipage}\\
%
%
%If the symmetry axes $\Phi _{e}, \Theta _{e}$ are aligned with the coordinate system, equation \ref{eq:Cdpsym} can be simplified to the relation given in equation \ref{eq:Cdpsymb} \cite{aeshs}.
%
%\begin{equation}
%C_{mn}(f) = C_{-mn}(f) \; \; \; \;  m = 2s; \; s =1, 2, 3...
%\label{eq:Cdpsymb}
%\end{equation}
%\myequations{Coefficients for dual plane symmetry}
%The symmetry factor for the Dual Plane symmetry is then given by equation \ref{eq:Sdps} \cite{aeshs}.
%
%
%\begin{equation}
%S_{dps} = 1-\frac{\sum_{n=1}^{N}\sum_{s=1}^{n/2}\left | (-1)^{2s}C_{(2s)n} - C_{(2s)n}\right |^{2} + \sum_{n=1}^{N}\sum_{s=0}^{n/2}\left | C_{(2s+1)n}\right |^{2}}{\sum_{n=0}^{N}\sum_{m=-n}^{n} C_{mn}^{2}}
%\label{eq:Sdps}
%\end{equation}
%\myequations{Symmetry factor for dual-plane symmetry}
%
%\vspace{0.1cm}
%
%\begin{minipage}{0.7\textwidth}
%\textbf{Baffle symmetry} is when the Device Under Test is mounted on a baffle (see figure \ref{fig:bafflesym}), and make half of the coefficient vanish according to the relation given by equation \ref{eq:Cbsym} \cite{aeshs}.
%
%\begin{equation}
%C_{mn} = 0 \; \; \; \; \; \; n-m \, \neq \, 2s; \: \: s \in \mathbb{Z}
%\label{eq:Cbsym}
%\end{equation}
%\myequations{Coefficients for baffle symmetry}
%The symmetry factor is then given by equation \ref{eq:Sbs} \cite{aeshs}.
%
%\begin{equation}
%S_{bs} = 1-\frac{\sum_{n=1}^{N}\sum_{s=0}^{n/2}\left | C_{(2s)n} \right |^{2}}{\sum_{n=0}^{N}\sum_{m=-n}^{n} C_{mn}^{2}}
%\label{eq:Sbs}
%\end{equation}
%\myequations{Symmetry factor for baffle symmetry}
%\end{minipage}
%\begin{minipage}{0.3\textwidth}
%\begin{center}
%	\includegraphics[width=0.95\textwidth]{Appendix/Baffle_Sym}
%    \captionsetup{hypcap=false}
%    \captionof{figure}{Baffle symmetry}
%    \label{fig:bafflesym}
%\end{center}
%\end{minipage}
%
%\vspace{0.1cm}
%
%\begin{minipage}{0.7\textwidth}
%\textbf{Single plane symmetry} is the most common in loudspeaker systems using 2 or more drivers. This symmetry can be between the left and right side of the DUT, as presented in figure \ref{fig:singplsym}; but also between the top and bottom of the DUT, shifting the symmetry plane by 90°. Only the coefficients on the left side have to be estimated. The coefficients of the right side are linked to the left side by the relation given by equation \ref{eq:spsym} \cite{aeshs}.
%
%\begin{equation}
%C_{mn} = C_{-mn}(f)R_{m}(f) \; \; \; \textup{with} \; \; m\geq 0; \; \; \; 0\leq n\leq N
%\label{eq:spsym}
%\end{equation}
%\myequations{Coefficients for single plane symmetry}
%It should be noted that depending of the type of symmetry (left/right or top/bottom), $R_{m}$ will change, modifying the relation between the coefficients. \\
%
%The symmetry factor is then given by equation \ref{eq:Ssps} \cite{aeshs}.
%
%\begin{equation}
%S_{sps} = 1-\frac{\sum_{n=1}^{N}\sum_{m=1}^{n}\left | (-1)^{m}(f) C_{-mn}-C_{mn} \right |^{2}}{\sum_{n=0}^{N}\sum_{m=-n}^{n} C_{mn}^{2}}
%\label{eq:Ssps}
%\end{equation}
%\myequations{Symmetry factor for single plane symmetry}
%\end{minipage}
%\begin{minipage}{0.3\textwidth}
%\begin{center}
%	\includegraphics[width=\textwidth]{Appendix/Single_Plane_Sym}
%    \captionsetup{hypcap=false}
%    \captionof{figure}{Single Plane symmetry}
%    \label{fig:singplsym}
%\end{center}
%\end{minipage}



% Symmetry

%\chapter[Equations for symmetry condition on complex coefficients of spherical wave expansion]{Equations for symmetry condition on complex \\ coefficients of spherical wave expansion}
%\label{chap:sym}
%
%\textit{\textbf{NOTE}: the presented equations are coming from the reference \cite{aeshs}. They are displayed in this report for a more comfortable reading experience and for reference, and I do not take any credits for the presented results.} \\
%
%\begin{minipage}{0.5\textwidth}
%\begin{center}
%	\includegraphics[scale=0.25]{Appendix/Spherical_Harmo}
%    \captionsetup{hypcap=false}
%    \captionof{figure}{Spherical Harmonics}
%    \label{fig:Sph_Harmo}
%\end{center}
%\end{minipage}
%\begin{minipage}{0.5\textwidth}
%\vspace{0.2cm}
%The solution of the three-dimensional wave equation in spherical coordinates can be described by a spherical wave expansion given by equation \ref{eq:sphwvexp}.The free-field transfer function of a loudspeaker can be expressed by only considering the direct sound (outgoing wave) in equation \ref{eq:sphwvexp} and is expressed in equation \ref{eq:ffHf}.\\
%
%Figure \ref{fig:Sph_Harmo} presents a visual representation of the first few spherical harmonics, and their orders (n,m) (\textit{all images presented in this chapter are courtesy of W. Klippel, from the presentation \cite{lect2018}}). \\~\\~\\~\\
%\end{minipage}
%
%\textbf{Influence of symmetry on coefficients} \\
%
%\begin{minipage}{0.7\textwidth}
%Rotational symmetry (see figure \ref{fig:rotsym}) is observed for many circular drivers and is the simplest type of symmetry as a lot of coefficients are vanishing according to the relation given by equation \ref{eq:Crotsym}.
%
%\begin{equation}
%C_{mn} = 0 \; \; \; \;  m \neq 0
%\label{eq:Crotsym}
%\end{equation}
%\myequations{Coefficients for rotational symmetry}
%All m $\neq$ 0 coefficients are vanishing, meaning that only m = 0, n = n+1 coefficients have to be estimated. \\
%
%Once the coefficients have been estimated, s symmetry factor S can be estimated. For rotation symmetry, $S_{rs}$ is given by equation \ref{eq:Srs}.
%
%\begin{equation}
%S_{rs} = 1-\frac{\sum_{n=1}^{N}\sum_{s=1}^{n}\left | C_{sn} \right |^{2}}{\sum_{n=0}^{N}\sum_{m=-n}^{n} C_{mn}^{2}}
%\label{eq:Srs}
%\end{equation}
%\myequations{Symmetry factor for rotational symmetry}
%
%\end{minipage}
%\begin{minipage}{0.3\textwidth}
%\begin{center}
%	\includegraphics[width=\textwidth]{Appendix/Rot_Sym}
%    \captionsetup{hypcap=false}
%    \captionof{figure}{Rotational symmetry}
%    \label{fig:rotsym}
%\end{center}
%\end{minipage}
%
%\begin{minipage}{0.7\textwidth}
%Dual plane symmetry (see figure \ref{fig:dualpsym}) is mostly observed in 2-way closed box systems, and make half of the coefficients vanish according to the relation given by equation \ref{eq:Cdpsym}. 
%
%\begin{equation}
%C_{mn}(f) = C_{-mn}(f)R_{m}(f) \; \; \; \;  m = 2s; \; s =1, 2, 3...
%\label{eq:Cdpsym}
%\end{equation}
%\myequations{Coefficients for dual plane symmetry}
%Where Rm represents a complex symmetry parameter, expressed in equation \ref{eq:Rm}.
%
%\begin{equation}
%R_{m}(f) = (-1)^{m+1}(\textup{sin}(m \phi _{s}(f)) + i\textup{cos}(m \phi _{s}(f)))^{2}
%\label{eq:Rm}
%\end{equation}
%\myequations{Complex symmetry parameter}
%With $\phi _{s}$ is the symmetry axis; and $m$ the suborder of the coefficient under study.
%\end{minipage}
%\begin{minipage}{0.3\textwidth}
%\begin{center}
%	\includegraphics[width=\textwidth]{Appendix/Dual_Plane_Sym}
%    \captionsetup{hypcap=false}
%    \captionof{figure}{Dual Plane symmetry}
%    \label{fig:dualpsym}
%\end{center}
%\end{minipage}\\
%
%
%If the symmetry axes $\Phi _{e}, \Theta _{e}$ are aligned with the coordinate system, equation \ref{eq:Cdpsym} can be simplified to the relation given in equation \ref{eq:Cdpsymb}.
%
%\begin{equation}
%C_{mn}(f) = C_{-mn}(f) \; \; \; \;  m = 2s; \; s =1, 2, 3...
%\label{eq:Cdpsymb}
%\end{equation}
%\myequations{Coefficients for dual plane symmetry}
%The symmetry factor for the Dual Plane symmetry is then given by equation \ref{eq:Sdps}.
%
%
%\begin{equation}
%S_{dps} = 1-\frac{\sum_{n=1}^{N}\sum_{s=1}^{n/2}\left | (-1)^{2s}C_{(2s)n} - C_{(2s)n}\right |^{2} + \sum_{n=1}^{N}\sum_{s=0}^{n/2}\left | C_{(2s+1)n}\right |^{2}}{\sum_{n=0}^{N}\sum_{m=-n}^{n} C_{mn}^{2}}
%\label{eq:Sdps}
%\end{equation}
%\myequations{Symmetry factor for dual-plane symmetry}
%
%\vspace{1cm}
%
%\begin{minipage}{0.7\textwidth}
%Baffle symmetry is when the Device Under Test is mounted on a baffle (see figure \ref{fig:bafflesym}), and make half of the coefficient vanish according to the relation given by equation \ref{eq:Cbsym}.
%
%\begin{equation}
%C_{mn} = 0 \; \; \; \; \; \; n-m \, \neq \, 2s; \: \: s \in \mathbb{Z}
%\label{eq:Cbsym}
%\end{equation}
%\myequations{Coefficients for baffle symmetry}
%The symmetry factor is then given by equation \ref{eq:Sbs}.
%
%\begin{equation}
%S_{bs} = 1-\frac{\sum_{n=1}^{N}\sum_{s=0}^{n/2}\left | C_{(2s)n} \right |^{2}}{\sum_{n=0}^{N}\sum_{m=-n}^{n} C_{mn}^{2}}
%\label{eq:Sbs}
%\end{equation}
%\myequations{Symmetry factor for baffle symmetry}
%\end{minipage}
%\begin{minipage}{0.3\textwidth}
%\begin{center}
%	\includegraphics[width=\textwidth]{Appendix/Baffle_Sym}
%    \captionsetup{hypcap=false}
%    \captionof{figure}{Baffle symmetry}
%    \label{fig:bafflesym}
%\end{center}
%\end{minipage}
%
%\vspace{1cm}
%
%\begin{minipage}{0.7\textwidth}
%Single plane symmetry is the most common in loudspeaker systems using 2 or more drivers. This symmetry can be between the left and right side of the DUT, as presented in figure \ref{fig:singplsym}; but also between the top and bottom of the DUT. Only the coefficients on the left side have to be estimated. The coefficients of the right side are linked to the left side by the relation given by equation \ref{eq:spsym}.
%
%\begin{equation}
%C_{mn} = C_{-mn}(f)R_{m}(f) \; \; \; \textup{with} \; \; m\geq 0; \; \; \; 0\leq n\leq N
%\label{eq:spsym}
%\end{equation}
%\myequations{Coefficients for single plane symmetry}
%It should be noted that depending of the type of symmetry (left/right or top/bottom), $R_{m}$ will change, modifying the relation between the coefficients. \\
%
%The symmetry factor is then given by equation \ref{eq:Ssps}.
%
%\begin{equation}
%S_{sps} = 1-\frac{\sum_{n=1}^{N}\sum_{m=1}^{n}\left | (-1)^{m}(f) C_{-mn}-C_{mn} \right |^{2}}{\sum_{n=0}^{N}\sum_{m=-n}^{n} C_{mn}^{2}}
%\label{eq:Ssps}
%\end{equation}
%\myequations{Symmetry factor for single plane symmetry}
%\end{minipage}
%\begin{minipage}{0.3\textwidth}
%\begin{center}
%	\includegraphics[width=\textwidth]{Appendix/Single_Plane_Sym}
%    \captionsetup{hypcap=false}
%    \captionof{figure}{Single Plane symmetry}
%    \label{fig:singplsym}
%\end{center}
%\end{minipage}